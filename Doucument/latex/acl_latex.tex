% This must be in the first 5 lines to tell arXiv to use pdfLaTeX, which is strongly recommended.
\pdfoutput=1
% In particular, the hyperref package requires pdfLaTeX in order to break URLs across lines.

\documentclass[11pt]{article}
\usepackage[dvipsnames]{xcolor}         % colors
% Remove the "review" option to generate the final version.
\usepackage[review]{acl}

% Standard package includes
\usepackage{times}
\usepackage{latexsym}

% For proper rendering and hyphenation of words containing Latin characters (including in bib files)
\usepackage[T1]{fontenc}
% For Vietnamese characters
% \usepackage[T5]{fontenc}
% See https://www.latex-project.org/help/documentation/encguide.pdf for other character sets

% This assumes your files are encoded as UTF8
\usepackage[utf8]{inputenc}

% This is not strictly necessary, and may be commented out,
% but it will improve the layout of the manuscript,
% and will typically save some space.
\usepackage{microtype}
\usepackage{hyperref}       % hyperlinks
\usepackage{url}            % simple URL typesetting
\usepackage{booktabs}       % professional-quality tables
\usepackage{amsfonts}       % blackboard math symbols
\usepackage{nicefrac}       % compact symbols for 1/2, etc.
\usepackage{microtype}      % microtypography

\usepackage{multirow}
\usepackage{lipsum}
\usepackage{verbatim}
\usepackage{amsmath}
\usepackage[toc,page,header]{appendix}
\usepackage{minitoc}
\usepackage{graphicx}

\usepackage{floatrow}
\usepackage{array}
\usepackage{subcaption}
\usepackage{hyperref}
\hypersetup{
    colorlinks=true,
    linkcolor=blue,
    filecolor=magenta,      
    urlcolor=blue,
    pdftitle={Overleaf Example},
    pdfpagemode=FullScreen,
    }
% If the title and author information does not fit in the area allocated, uncomment the following
%
%\setlength\titlebox{<dim>}
%
% and set <dim> to something 5cm or larger.

\title{Instruction-tuned LLMs}

% Author information can be set in various styles:

\author{%
Yuansheng Ni \\
  Zhejiang University\\
  %Boston, MA 02115 \\
  \texttt{yuansheng.ni@zju.edu.cn} \\
  \And 
  Hui Shen \\
  The Ohio State University\\
  %Boston, MA 02115 \\
  \texttt{shen.1780@osu.edu}
}

\begin{document}
\maketitle
\begin{abstract}
  \emph{Instruction fine-tuning} has recently emerged as a promising approach for improving the zero-shot capabilities of Large Language Models (LLMs) on new tasks.
  This technique has shown particular strength in improving the performance of modestly sized LLMs, sometimes inducing performance competitive with much larger model variants.
  The focus is on the robustness of instruction-tuned Large Language Models (LLMs) to unseen instructions and unseen tasks.
  We conducted an exploration on four models including Alpaca, Vicuna, WizardLM, and Traditional Task-oriented Models using real-world relation extraction datasets as case studies.
  We carried out a comprehensive evaluation of these instruction-following large language models which have been tuned based on open-domain instructions and task-oriented instructions.
  The central discussion is on how to enhance their robustness to natural language variation.
  We observed that xxxx  .
  This consistently improves the robustness of the instruction-tuned models.

\end{abstract}

\section{Introduction}

\begin{figure}[h]
  \centering
  %\includegraphics[width=140mm]{images/fig1-redux.pdf}
  \caption{How well do models trained on instruction-tuning datasets generalize to novel instructions (unobserved in training)? Our analysis suggests that they do not do so very well. Above we show a case where pairing an example with an observed instruction yields the correct output, while providing a distinct but semantically equivalent instruction produces an incorrect response.
    We propose and evaluate a simple method that improves this.}
  %Flan \cite{chung2022scaling} is a meta-dataset comprising a total of 1.8k tasks. We ask: How well do models trained on this collection of instructions generalize to novel instructions and tasks (unobserved in training), and how can we improve their ability to do so?} %We study how far can the models tuned with such instruction collection be generalized into other unobserved instructions and tasks.}
  \label{fig:main_fig}
\end{figure}


Large Language Models (LLMs) have come to dominate NLP, in part because they enable zero- and few-shot adaptation to new tasks via \emph{prompting} \cite{brown2020language, chowdhery2022palm, hoffmann2022training, zeng2022glm}.
Recent work has demonstrated the promise of fine-tuning such models with natural language instructions.
Such \emph{instruction-tuning} improves LLM performance in zero- and few-shot settings, sometimes dramatically, especially for ``mid-sized'' models \cite{chung2022scaling, ouyang2022training}.
For example, on some benchmarks the instruction-tuned Flan-T5-XL (3B parameters) \cite{chung2022scaling} outperforms GPT-3 (175B), despite being dramatically smaller.
Furthermore, LLaMa-7B \cite{touvron2023llama}---after being fine-tuned on large-scale corpora on the Alpaca \cite{alpaca} instruction set---outperforms GPT-3 across a range of NLP benchmarks.
%How does instruction-tuning work? Several efforts have investigated

These empirical successes have motivated efforts to curate instruction-augmented task collections for meta-learning \cite{wang2022benchmarking, wei2021finetuned, wei2021finetuned}, and research into improving instruction-tuning \cite{longpre2023flan, xu2022multiinstruct, sanh2021multitask}. %Nonetheless, it is not clear how far the ability gained from instruction tuning 
In this work we investigate how robust instruction-tuned models are.
%can be generalized to domains, tasks, and instructions that are \textbf{\textit{unobserved}} during instruction fine-tuning. In this work, we investigate the \emph{robustness} of instructions. 
More specifically, we ask: How sensitive are instruction-tuned LMs to shifts in instruction phrasings at test time?
This is particularly important given that the primary motivation of instruction tuning is to facilitate zero-shot adaptation via natural language instruction: If models are overly sensitive to the particular phrasing of a task instruction it may greatly limit their utility in practice.

%the shifts of instruction and task distribution from the training stage?



Prior work---reviewed at length in Section \ref{section:related-work}---has established that LLMs do not seem to intuitively ``understand'' prompts \cite{webson2021prompt,jang2023can, zhang2023aligning}, but these efforts did not consider instruction-tuned models specifically.
Recent, contemporaneous work to ours \cite{gu2023robustness} investigated the robustness of instruction-tuned models, and found that instruction-tuned T5 \cite{raffel2020exploring} is robust to instruction perturbations in few-shot settings, but less so in zero-shot application.
We contribute a more in-depth analysis of this phenomena across a much wider set of instruction-tuned models and benchmarks.
We also introduce and evaluate a method for improving the robustness of such models, with promising results.
% by imposing an objective encouraging LLMs to induce similar representations for semantically equivalent instructions.

%To address this research question, 
More specifically, we collect a relatively large set of task instructions manually composed by NLP researchers; these are valid instructions but distinct from those found in the Flan collection.
%\textbf{\textit{unobserved}} during instruction fine-tuning. 
We then assess the performance of LLMs fine-tuned on the Flan collection instruction set when given these novel instructions on two benchmarks: \textsc{MMLU} \cite{hendrycks2020measuring} and \textsc{BBL} \cite{srivastava2022beyond}.
%We perform inferences on MMLU \cite{hendrycks2020measuring} and BBL \cite{srivastava2022beyond} with LMs that are instruction-tuned with the Flan collection. 
%It is observed that there is an 
We find that using novel instructions in zero-shot application degrades accuracy considerably (Figure \ref{fig:main_fig} illustrates this).
For example, comparing the performance of Flan-T5 XXL when using (a) instructions that were seen in training to (b) semantically equivalent but unobserved in training, we observe a 6.9 point drop in absolute performance on average across large benchmarks.

We thoroughly analyze the robustness of instruction-tuned LLMs across three model "families" in a comprehensive and detailed manner.

The detailed contributions of this paper are as follows:


Our {\bf main contributions} are summarized as follows. (1) We perform a comprehensive and in-depth analysis of the robustness of instruction-tuned LLMs across three ``families'' of such models (Flan-T5 \cite{wei2021finetuned}, Alpaca \cite{alpaca}, and T0 \cite{sanh2021multitask}) using large benchmarks \cite{hendrycks2020measuring,srivastava2022beyond}.
For this we collect a large set of new task instructions manually composed by researchers in NLP; we will release this dataset to facilitate additional work on instruction robustness. We observe substantial performance degradation when using ``novel'' (unseen in training) instructions.
(2) We propose a simple method to improve robustness by imposing an objective encouraging LLMs to induce similar representations for semantically equivalent instructions.
We find that this consistently improves the performance realized when using novel but appropriate task instructions.


%when using semantically appropriate instructions that were ufor Flan-T5-11B we observe an average accuracy degradation across
% $\textbf{4.93}$/$\textbf{2.34}$ while instructing the language model (Flan-T5-11B) with paraphrased instructions and manually written instructions on MMLU and $\textbf{3.74}$/$\textbf{15.29}$ on BBL. Additionally, we discover that on completely unobserved tasks, the instructions written by domain experts are even outperformed by observed instructions that are completely unrelated.

% TODO add note about adversarial result and then the soft propmting method; maybe also something about distances of represntations etc

%We discovered that in most cases, this gap in performance between observed instructions and unobserved instructions is rapidly narrowing down after an example (one-shot) is provided. By observing the distance over LMs' latent space, we discovered that in-context learning drags the distribution of instructions closer and hence increases the robustness of LMs significantly. To validate this discovery, we ...


\begin{comment}
The most current Large Language Models (LLMs), such as PaLM \cite{chowdhery2022palm}, Chinchilla \cite{hoffmann2022training}, and GLM-130B \cite{zeng2022glm}, have achieved remarkable performance on many NLP tasks and their downstream applications.
These LLMs are known for their strong generalizability and in-context learning brought by the emergent ability.
It has been recently discovered that further fine-tuning these models with instructions yields even better zero-shot and few-shot performance on unseen tasks \cite{chung2022scaling, ouyang2022training}.
This procedure is commonly referred to as instruction fine-tuning.
With instruction tuning, models like Flan-T5-XXL can even outperform its counterpart GPT-3, which is 16 times larger in size \cite{chung2022scaling}.
\end{comment}

\begin{comment}
Consequently, many have carried out research on the efficacy of instruction fine-tuning \cite{longpre2023flan, xu2022multiinstruct, sanh2021multitask} and constructed open-source resources for multitask fine-tuning with instructions \cite{wang2022benchmarking, wei2021finetuned, wei2021finetuned}. Adequate evidence shows that instruction fine-tuning could significantly unlock the knowledge-understanding capability of LMs from pre-training and further enhance its generalizability. However, the robustness and sensitivity of this new paradigm have not been sufficiently studied.
\end{comment}

%In this paper, we ...

\begin{comment}
\paragraph{How robust are instruction-tuned LLMs?}

%The surge of downstream applications has raised awareness of the other important aspects of instruction-tuned language models besides best performances. 
Instruction-tuning has shown promise in improving zero-shot performance of LLMs, especially for comparatively small models \cite{alpaca,longpre2023flan,sanh2021multitask}.
Indeed, careful instruction-tuning of ``smaller'' models like T5-XL (3B parameters) can result in zero-shot performance that is competitive with---or even better than---much larger models \cite{longpre2023flan}.
Realizing such functionality with modestly sized models is an important and timely research goal, and instruction-tuning has emerged as a promising means to this end.

The promise of such models lies in their ability to perform novel tasks specified via natural language instructions.
It is therefore important that models are robust to variations in instruction phrasings: Semantically equiavelent instructions should yield comparable results.
However, we find that instruction-tuned LLMs are (sometimes very) sensitive to changes in phrasings, with novel instructions (i.e., instructions unobserved in training) leading to substantially degraded performance.

%For a specific task, a \textbf{robust} model, after instruction-tuning, ought to perform equally well when given two semantically equivalent instructions
\end{comment}
\begin{comment}
\begin{figure}%[h]
  \centering
  \includegraphics[scale=0.45]{images/data_cropped.pdf}
  \caption{To evaluate the generalization capabilities of LLMs on unobserved instruction and tasks, we collect additional instruction templates in two ways: (1) Automatically paraphrasing from the original Flan collection, and (2) Enlisting NLP researchers to manually compose instructions for tasks. In total, we collected 1206 instruction templates which we use to evaluate the robustness of instruction-tuned models.}
  %In total, we collected a set of 1206 instruction templates to compare with the performance of instructions observed during training }
  \label{fig:data}
\end{figure}
\end{comment}

\begin{comment}

\begin{figure}[h]
  \centering
  \includegraphics[scale=0.5]{images/robustness_cropped.pdf}
  \caption{In this example, the observed instruction is the \textit{Task 1286} from NIV2 \cite{wang2022benchmarking} using \textit{Template 6}, while the unobserved instruction shown was composed by an NLP practitioner specifically for MMLU \cite{hendrycks2020measuring}. The model fails in this latter case, despite the instruction being valid.}
  \label{fig:my_label}
\end{figure}
\end{comment}

\section{Related Work}
\label{section:related-work}

\paragraph{Multitask learning and instruction-tuning}

%Before the instruction era, previous works of multi-task fine-tuning focused on bettering the model's NLU ability by unifying several downstream tasks into one and fine-tuning over the unified corpus 
%Prior to the introduction of \emph{instruction-tuning}, 
Training a single text-to-text model capable of providing responses to arbitrary queries has been an aspiration in NLP for at least half a decade. 
%pre-dates the current wave of LLMs and modern prompting and instructing strategies. 
%Prior to modern prompting and instructing strategies, 
For example, prior to modern prompting and instructing strategies, there were efforts to unify disparate tasks by reframing them as instances of general \emph{question answering} \cite{mccann2018natural, khashabi2020unifiedqa, keskar2019unifying}. 
%After that, many contemporary works carry out meta-dataset resources for multi-task tuning with instructions to unlock the hidden knowledge learned through large-scale unsupervised learning, the representative works are: Flan 2021 collection \cite{wei2021finetuned}, Natural Instructions \cite{mishra2021cross} and T0 \cite{sanh2021multitask}. 
More recent efforts have focussed on compiling and fine-tuning LLMs on corpora comprising diverse tasks with associated natural language instructions \cite{wei2021finetuned,mishra2021cross,sanh2021multitask}; we refer to this strategy as instruction-tuning. 
One example of this is {\tt Super-NaturalInstructions} \cite{wang2022benchmarking}, which compiles over 1600 tasks and enriches these with both instructions and negative examples. %to enrich the instruction collection. On top of that, t
Similarly, the recently released OPT-IML Bench \cite{iyer2022opt} comprises 2000 NLP tasks. %covering a range of categories.
The Flan 2022 task collection \cite{longpre2023flan} additionally features \emph{Chain-of-Thought} (CoT) style ``reasoning'' chains in instruction templates; the authors show that including these (as well as zero-shot examples and ``input inversions'') during instruction fine-tuning yields improvements on held-out tasks. 
%he new Flan collection further incorporates Chain-of-Thought (CoT) reasoning into the instruction templates and shows that it benefits instruction fine-tuning. 

These meta-resources---collections of instructions, tasks, and samples---have facilitated the training of instruction-tuned model families such as Flan-T5, Flan-PaLM \cite{chung2022scaling}, and OPT-IML \cite{iyer2022opt}.\footnote{Somewhat confusingly, in the case of FLAN and OPT, the corpora (i.e., benchmarks comprising tasks and instructions) and LLMs fine-tuned using them are both referred to with the associated acronym as prefix: For instance, Flan-T5 denotes a T5 \cite{raffel2020exploring} variant fine-tuned with the Flan collection.}
Results have been encouraging; fine-tuning LLMs to follow instructions provides clear and consistent gains across models, and, perhaps most exciting, enables relatively ``small'' ($\sim$10B) LLMs to achieve near SOTA performance comparable to massive ($\sim$175B) models \cite{alpaca}. 
This has motivated interest in characterizing how instructions help models, and developing techniques to further improve instruction-tuning; we review recent efforts related to these two research threads below. 

%Subsequently, with all these resources available many instruction-tuned models like Flan-T5, Flan-PaLM \cite{chung2022scaling}, and OPT-IML \cite{iyer2022opt} are trained and achieved state-of-the-art performance on many downstream tasks.
\paragraph{Evaluating prompting and instruction capabilities}
%Inherited from the prompt learning paradigm, many works aim to evaluate the hidden mechanism of instruction tuning and its potential flaws. 
Instructions may be seen as a special sort of model prompting, which a few recent efforts have critically evaluated. 
For example, Webson and Pavlick ask whether models meaningfully ``understand'' prompts \cite{webson2021prompt}, finding that they largely do not: %looks at the choices of prompt for inference over instruction fine-tuned model and found that the 
Performance is often unaffected when irrelevant and misleading prompts are provided. 
In follow up work, Jang \emph{et al.}
\cite{jang2023can} evaluates performance on negated prompts, observing an ``inverse-scaling'' phenomenon in which larger models perform worse in this case.
%studies the negated prompt and found that it inversely scaled with the size of the model. 

Other work has attempted to characterize how and when \emph{in-context learning} (ICL)---i.e., including a few examples in prompts---works \cite{min2022rethinking,wang2023large,dai2022can,akyurek2022learning,yu2022alert}. 
ICL is a form of prompting orthogonal to the present effort, as we are primarily interested in the zero-shot adaptability of instruction-tuned LLMs.

%With respect to instruction tuned models specifically, Yu \emph{et al.} \cite{yu2022alert} show that instruction tuning enhances zero-shot performance of the model at the cost of harming its ability to follow various instructions. 
%\cite{min2022rethinking} examine the performance of instruction-tuned models on classification tasks and discover that altering the label space in the instruction has a marginal impact on the performance.

In work contemporaneous to ours, Gu \emph{et al.} \cite{gu2023robustness} investigated how robust instruction-tuned models are to instruction perturbations (e.g., dropping words) and paraphrasings. 
They found that models are relatively robust when given examples (i.e., in few-shot settings), but quite sensitive when used zero-shot; this is qualitatively in line with our findings.
Our work differs in important way from this coincident research: (1) We provide a much more comprehensive analysis of robustness; Gu \emph{et al.} considered \emph{only} T5 instruction-tuned on a single instruction dataset, whereas we evaluate three LLMs (and different sizes of each) using five instruction tuning datasets, and we evaluate using over 80 test tasks in all (Gu \emph{et al.} considered only 12). (2) We propose and evaluate a new approach to \emph{improving} the robustness of instruction-tuned models; Gu \emph{et al.} offered no mechanism to improve robustness. 


\paragraph{Improving instruction-tuning}
%Most of the works aim to improve the performance of instruction-tuned models are focusing on two threads. 
Past work has also sought to improve instruction-tuning in various ways.
One means to do so is to instruction tune based on human feedback \cite{ouyang2022training, glaese2022improving, bai2022training, nakano2021webgpt, zhang2023wisdom}. 
This tends to improve open-ended model responses but degrade performance on downstream tasks. %improves the model's open-ended task performance at the cost of NLP tasks performance degradation. 
Another strategy is to leverage existing resources to automatically generate instruction-tuning datasets at scale. 
For example, Wang \emph{et al.} \cite{wang2022self} use LLMs to generate instructions, inputs, and outputs and use these to improve their own instruction-following capabilities. 
%improves the instruction-following capabilities of LLMs by bootstrapping their own generations to train the model.
In a similarly meta vein, Zhou and colleagues \cite{zhou2022large}  propose using LLMs to engineer prompts. 
%regards instructions as program to perform text-to-structure generation with LLM. 
Finally, Ye \emph{et al.} \cite{ye2022guess} propose ``flipping'' the standard task by tasking LLMs with generating \emph{instructions}, given an input and label. 
%trains the LM to produce instructions given the input and labels. 
\section{Instruction Datasets}

\subsection{Evaluation Benchmarks}

%We perform the evaluation of instruction-tuned models 
We evaluate a set of instruction-tuned models on two large benchmarks: \textsc{MMLU} \cite{hendrycks2020measuring} and \textsc{Big-Bench} \cite{srivastava2022beyond}. \textsc{MMLU} is a multiple-choice question-answering benchmark comprising 57 tasks that require expert knowledge.
\textsc{Big-Bench} is a collaboratively built benchmark containing 204 diverse tasks from various domains; here %we use the 18 task 
consider the \textsc{Big-Bench Lite} subset, and we include only QA, multi-class, and binary classification tasks, yielding 18 tasks from in all. %in \textsc{Big-Bench}.

%More specifically, We conduct our experiment over all 57 tasks on MMLU. An 18 tasks subset of \textsc{Big-Bench Lite} consists of all QA, multi-class, and binary classification tasks.

\subsection{Collecting New Instructions from NLP Researchers}
\label{section:new-instructions}

We aim to evaluate instruction-tuned models when they are provided instructions which are semantically equivalent to, but superficially different from, those with which they were trained.
To this end, we enlist NLP researchers (graduate students) to compose novel instructions for the tasks considered; these particular instruction phrasings were therefore \emph{unobserved} during instruction fine-tuning. 

%For each instruction-tuned language model we  we collect a set of instructions unobserved to the model by expert annotation, and we collect a set of instructions observed during training from the original instruction-tuning collection.

% TODO: to have some small tables here to show the stats

%\subsubsection{Unobserved Instruction}

%We perform large-scale crowd-sourcing from 
More specifically, we recruited 36 NLP graduate students working in NLP.
All had at least some experience with instruction-tuned models and the downstream tasks included in the evaluation benchmarks. 
For each of the 18 tasks in \textsc{BBL} and all tasks in \textsc{MMLU}, we asked 12 graduate students to write one (distinct) instruction they would use for zero-shot inference with an instruction-tuned model. 
%To ensure fairness, the information on models is omitted to avoid having priors to fit the pattern of the specific model. 
%The detailed instruction collection process can be seen in Appendix A.
We provide details on this instruction collection process in Appendix A. 
We will release all 319 instructions acquired for this work to ensure the reproducibility of this work and to facilitate further research on instruction-tuned model robustness. 
% We treat 57 tasks of MMLU as a whole (general QA template). Don't know how to make the word clearer

%\subsubsection{Observed Instruction}


% could be generally applied to one of ``multiple-choice QA'', ``binary label classification'', and ``m
%``multi-label classification" tasks. 

% TODO: to have some small tables here to show the stats






\input{sections/04_experiment.tex}
\section{Conclusions}
\label{section:conclusions}
\vspace{-.35em}

Instruction-tuned LLMs have emerged as a promising means of achieving zero-shot performance with smaller models that is competitive to, and sometimes even better than, that observed using much larger LLMs \cite{longpre2023flan,alpaca}.
In this work we empirically characterized the \emph{robustness} of such models with respect to instruction rephrasings.
In particular, we collected manually composed instructions from 36 graduate students in NLP across 75 tasks, and we evaluated different families of instruction-tuned LLMs (Flan, Alpaca, and T0) when provided observed and unobserved instructions (seen in training and not, respectively). %(i.e., seen in training) and unobserved (manual instructions we collected).
We found that using the latter consistently degrades model performance, indicating that models are unduly sensitive to instruction phrasings.

We then proposed a simple mechanism intended to improve the robustness of instruction-tuned LLMs.
This approach entails introducing an additional loss term that penalizes the model for inducing dissimilar distributions over output tokens when using (a) paraphrased instructions as opposed to (b) a reference instruction for the same task.
We found that training under this objective consistently (though modestly) improves results, and in particular mitigates the degradation observed when previously unobserved instructions are used.

\section{Limitations}
\label{section:limitations}
\vspace{-.35em}

This work has important limitations: For example we only evaluated ``mid-sized'' models (<20B parameters), it is unclear if our findings would generalize to much larger instruction-tuned models. (However, we note that instruction tuning has been most promising for smaller models.)
We also restricted our evaluation to three task types: QA and multi-class and binary classification.

\vspace{0.1em}
\noindent{\bf Ethics} This work does not have an explicit ethical dimension, but we acknowledge that all LLMs are likely to encode problematic biases; it is unclear how instruction-tuning might interact with these.

\section{Acknowledgments}

This work was supported by the National Science Foundation (NSF) grant 1901117.

We thank Jay DeYoung and Alberto Mario Ceballos Arroyo for their advice and feedback on the paper.
We also thank Alberto Mario Ceballos Arroyo, Arnab Sen Sharma, Bowen Zhao, Eric Todd, Hanming Li, Hiba Ahsan, Hye Sun Yun, Shulin Cao, Jay DeYoung, Jered McInerney, Ji Qi, Jifan Yu, Jize Jiang, Kaisheng Zeng, Koyena Pal, Kundan Krishna, Linxiao Nie, Hailong Jin, Jinxin Matthew Liu, Millicent Li, Monica Munnangi, Nikhil Prakash, Pouya Pezeshpour, Sanjana Ramprasad, Sarthak Jain, Shangqing Tu, Somin Wadhwa, Tingjian Zhang, Hao Wesley Peng, Xiaozhi Wang, Xingyu Lu, Xin Lv, Zijun Yao for providing manually written instructions.




\end{table*}



\section{Preamble}


Set the title and author using \verb|\title| and \verb|\author|. Within the author list, format multiple authors using \verb|\and| and \verb|\And| and \verb|\AND|; please see the \LaTeX{} source for examples.

By default, the box containing the title and author names is set to the minimum of 5 cm. If you need more space, include the following in the preamble:
\begin{quote}
  \begin{verbatim}
\setlength\titlebox{<dim>}
\end{verbatim}
\end{quote}
where \verb|<dim>| is replaced with a length. Do not set this length smaller than 5 cm.

\section{Document Body}

\subsection{Footnotes}

Footnotes are inserted with the \verb|\footnote| command.\footnote{This is a footnote.}



Table~\ref{citation-guide} shows the syntax supported by the style files.
We encourage you to use the natbib styles.
You can use the command \verb|\citet| (cite in text) to get ``author (year)'' citations, like this citation to a paper by \citet{Gusfield:97}.
You can use the command \verb|\citep| (cite in parentheses) to get ``(author, year)'' citations \citep{Gusfield:97}.
You can use the command \verb|\citealp| (alternative cite without parentheses) to get ``author, year'' citations, which is useful for using citations within parentheses (e.g. \citealp{Gusfield:97}).

\subsection{References}

\nocite{Ando2005,andrew2007scalable,rasooli-tetrault-2015}

The \LaTeX{} and Bib\TeX{} style files provided roughly follow the American Psychological Association format.
If your own bib file is named \texttt{custom.bib}, then placing the following before any appendices in your \LaTeX{} file will generate the references section for you:
\begin{quote}
  \begin{verbatim}
\bibliography{custom}
\end{verbatim}
\end{quote}

You can obtain the complete ACL Anthology as a Bib\TeX{} file from \url{https://aclweb.org/anthology/anthology.bib.gz}.
To include both the Anthology and your own .bib file, use the following instead of the above.
\begin{quote}
  \begin{verbatim}
\bibliography{anthology,custom}
\end{verbatim}
\end{quote}

Please see Section~\ref{sec:bibtex} for information on preparing Bib\TeX{} files.

\subsection{Appendices}

Use \verb|\appendix| before any appendix section to switch the section numbering over to letters. See Appendix~\ref{sec:appendix} for an example.

\section*{Acknowledgements}

% Entries for the entire Anthology, followed by custom entries
\bibliography{anthology,custom}

\appendix

\section{Example Appendix}
\label{sec:appendix}

This is an appendix.

\end{document}
