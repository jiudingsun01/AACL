% This must be in the first 5 lines to tell arXiv to use pdfLaTeX, which is strongly recommended.
\pdfoutput=1
% In particular, the hyperref package requires pdfLaTeX in order to break URLs across lines.

\documentclass[11pt]{article}
\usepackage[dvipsnames]{xcolor}         % colors
% Remove the "review" option to generate the final version.
\usepackage[review]{acl}

% Standard package includes
\usepackage{times}
\usepackage{latexsym}

% For proper rendering and hyphenation of words containing Latin characters (including in bib files)
\usepackage[T1]{fontenc}
% For Vietnamese characters
% \usepackage[T5]{fontenc}
% See https://www.latex-project.org/help/documentation/encguide.pdf for other character sets

% This assumes your files are encoded as UTF8
\usepackage[utf8]{inputenc}

% This is not strictly necessary, and may be commented out,
% but it will improve the layout of the manuscript,
% and will typically save some space.
\usepackage{microtype}
\usepackage{hyperref}       % hyperlinks
\usepackage{url}            % simple URL typesetting
\usepackage{booktabs}       % professional-quality tables
\usepackage{amsfonts}       % blackboard math symbols
\usepackage{nicefrac}       % compact symbols for 1/2, etc.
\usepackage{microtype}      % microtypography

\usepackage{multirow}
\usepackage{lipsum}
\usepackage{verbatim}
\usepackage{amsmath}
\usepackage[toc,page,header]{appendix}
\usepackage{minitoc}
\usepackage{graphicx}

\usepackage{floatrow}
\usepackage{array}
\usepackage{subcaption}
\usepackage{hyperref}
\hypersetup{
    colorlinks=true,
    linkcolor=blue,
    filecolor=magenta,      
    urlcolor=blue,
    pdftitle={Overleaf Example},
    pdfpagemode=FullScreen,
    }
% If the title and author information does not fit in the area allocated, uncomment the following
%
%\setlength\titlebox{<dim>}
%
% and set <dim> to something 5cm or larger.

\title{Instruction-tuned LLMs}

% Author information can be set in various styles:

\author{%
Yuansheng Ni \\
  Zhejiang University\\
  %Boston, MA 02115 \\
  \texttt{yuansheng.ni@zju.edu.cn} \\
  \And 
  Hui Shen \\
  The Ohio State University\\
  %Boston, MA 02115 \\
  \texttt{shen.1780@osu.edu}
}

\begin{document}
\maketitle
\begin{abstract}
  \emph{Instruction fine-tuning} has recently emerged as a promising approach for improving the zero-shot capabilities of Large Language Models (LLMs) on new tasks.
  This technique has shown particular strength in improving the performance of modestly sized LLMs, sometimes inducing performance competitive with much larger model variants.
  The focus is on the robustness of instruction-tuned Large Language Models (LLMs) to unseen instructions and unseen tasks.
  We conducted an exploration on four models including Alpaca, Vicuna, WizardLM, and Traditional Task-oriented Models using real-world relation extraction datasets as case studies.
  We carried out a comprehensive evaluation of these instruction-following large language models which have been tuned based on open-domain instructions and task-oriented instructions.
  The central discussion is on how to enhance their robustness to natural language variation.
  We observed that xxxx  .
  This consistently improves the robustness of the instruction-tuned models.

\end{abstract}

\section{Introduction}

\begin{figure}[h]
  \centering
  %\includegraphics[width=140mm]{images/fig1-redux.pdf}
  \caption{How well do models trained on instruction-tuning datasets generalize to novel instructions (unobserved in training)? Our analysis suggests that they do not do so very well. Above we show a case where pairing an example with an observed instruction yields the correct output, while providing a distinct but semantically equivalent instruction produces an incorrect response.
    We propose and evaluate a simple method that improves this.}
  %Flan \cite{chung2022scaling} is a meta-dataset comprising a total of 1.8k tasks. We ask: How well do models trained on this collection of instructions generalize to novel instructions and tasks (unobserved in training), and how can we improve their ability to do so?} %We study how far can the models tuned with such instruction collection be generalized into other unobserved instructions and tasks.}
  \label{fig:main_fig}
\end{figure}


Large Language Models (LLMs) have come to dominate NLP, in part because they enable zero- and few-shot adaptation to new tasks via \emph{prompting} \cite{brown2020language, chowdhery2022palm, hoffmann2022training, zeng2022glm}.
Recent work has demonstrated the promise of fine-tuning such models with natural language instructions.
Such \emph{instruction-tuning} improves LLM performance in zero- and few-shot settings, sometimes dramatically, especially for ``mid-sized'' models \cite{chung2022scaling, ouyang2022training}.
For example, on some benchmarks the instruction-tuned Flan-T5-XL (3B parameters) \cite{chung2022scaling} outperforms GPT-3 (175B), despite being dramatically smaller.
Furthermore, LLaMa-7B \cite{touvron2023llama}---after being fine-tuned on large-scale corpora on the Alpaca \cite{alpaca} instruction set---outperforms GPT-3 across a range of NLP benchmarks.
%How does instruction-tuning work? Several efforts have investigated

These empirical successes have motivated efforts to curate instruction-augmented task collections for meta-learning \cite{wang2022benchmarking, wei2021finetuned, wei2021finetuned}, and research into improving instruction-tuning \cite{longpre2023flan, xu2022multiinstruct, sanh2021multitask}. %Nonetheless, it is not clear how far the ability gained from instruction tuning 
In this work we investigate how robust instruction-tuned models are.
%can be generalized to domains, tasks, and instructions that are \textbf{\textit{unobserved}} during instruction fine-tuning. In this work, we investigate the \emph{robustness} of instructions. 
More specifically, we ask: How sensitive are instruction-tuned LMs to shifts in instruction phrasings at test time?
This is particularly important given that the primary motivation of instruction tuning is to facilitate zero-shot adaptation via natural language instruction: If models are overly sensitive to the particular phrasing of a task instruction it may greatly limit their utility in practice.

%the shifts of instruction and task distribution from the training stage?



Prior work---reviewed at length in Section \ref{section:related-work}---has established that LLMs do not seem to intuitively ``understand'' prompts \cite{webson2021prompt,jang2023can, zhang2023aligning}, but these efforts did not consider instruction-tuned models specifically.
Recent, contemporaneous work to ours \cite{gu2023robustness} investigated the robustness of instruction-tuned models, and found that instruction-tuned T5 \cite{raffel2020exploring} is robust to instruction perturbations in few-shot settings, but less so in zero-shot application.
We contribute a more in-depth analysis of this phenomena across a much wider set of instruction-tuned models and benchmarks.
We also introduce and evaluate a method for improving the robustness of such models, with promising results.
% by imposing an objective encouraging LLMs to induce similar representations for semantically equivalent instructions.

%To address this research question, 
More specifically, we collect a relatively large set of task instructions manually composed by NLP researchers; these are valid instructions but distinct from those found in the Flan collection.
%\textbf{\textit{unobserved}} during instruction fine-tuning. 
We then assess the performance of LLMs fine-tuned on the Flan collection instruction set when given these novel instructions on two benchmarks: \textsc{MMLU} \cite{hendrycks2020measuring} and \textsc{BBL} \cite{srivastava2022beyond}.
%We perform inferences on MMLU \cite{hendrycks2020measuring} and BBL \cite{srivastava2022beyond} with LMs that are instruction-tuned with the Flan collection. 
%It is observed that there is an 
We find that using novel instructions in zero-shot application degrades accuracy considerably (Figure \ref{fig:main_fig} illustrates this).
For example, comparing the performance of Flan-T5 XXL when using (a) instructions that were seen in training to (b) semantically equivalent but unobserved in training, we observe a 6.9 point drop in absolute performance on average across large benchmarks.

Our {\bf main contributions} are summarized as follows. (1) We perform a comprehensive and in-depth analysis of the robustness of instruction-tuned LLMs across three ``families'' of such models (Flan-T5 \cite{wei2021finetuned}, Alpaca \cite{alpaca}, and T0 \cite{sanh2021multitask}) using large benchmarks \cite{hendrycks2020measuring,srivastava2022beyond}.
For this we collect a large set of new task instructions manually composed by researchers in NLP; we will release this dataset to facilitate additional work on instruction robustness. We observe substantial performance degradation when using ``novel'' (unseen in training) instructions.
(2) We propose a simple method to improve robustness by imposing an objective encouraging LLMs to induce similar representations for semantically equivalent instructions.
We find that this consistently improves the performance realized when using novel but appropriate task instructions.


%when using semantically appropriate instructions that were ufor Flan-T5-11B we observe an average accuracy degradation across
% $\textbf{4.93}$/$\textbf{2.34}$ while instructing the language model (Flan-T5-11B) with paraphrased instructions and manually written instructions on MMLU and $\textbf{3.74}$/$\textbf{15.29}$ on BBL. Additionally, we discover that on completely unobserved tasks, the instructions written by domain experts are even outperformed by observed instructions that are completely unrelated.

% TODO add note about adversarial result and then the soft propmting method; maybe also something about distances of represntations etc

%We discovered that in most cases, this gap in performance between observed instructions and unobserved instructions is rapidly narrowing down after an example (one-shot) is provided. By observing the distance over LMs' latent space, we discovered that in-context learning drags the distribution of instructions closer and hence increases the robustness of LMs significantly. To validate this discovery, we ...


\begin{comment}
The most current Large Language Models (LLMs), such as PaLM \cite{chowdhery2022palm}, Chinchilla \cite{hoffmann2022training}, and GLM-130B \cite{zeng2022glm}, have achieved remarkable performance on many NLP tasks and their downstream applications.
These LLMs are known for their strong generalizability and in-context learning brought by the emergent ability.
It has been recently discovered that further fine-tuning these models with instructions yields even better zero-shot and few-shot performance on unseen tasks \cite{chung2022scaling, ouyang2022training}.
This procedure is commonly referred to as instruction fine-tuning.
With instruction tuning, models like Flan-T5-XXL can even outperform its counterpart GPT-3, which is 16 times larger in size \cite{chung2022scaling}.
\end{comment}

\begin{comment}
Consequently, many have carried out research on the efficacy of instruction fine-tuning \cite{longpre2023flan, xu2022multiinstruct, sanh2021multitask} and constructed open-source resources for multitask fine-tuning with instructions \cite{wang2022benchmarking, wei2021finetuned, wei2021finetuned}. Adequate evidence shows that instruction fine-tuning could significantly unlock the knowledge-understanding capability of LMs from pre-training and further enhance its generalizability. However, the robustness and sensitivity of this new paradigm have not been sufficiently studied.
\end{comment}

%In this paper, we ...

\begin{comment}
\paragraph{How robust are instruction-tuned LLMs?}

%The surge of downstream applications has raised awareness of the other important aspects of instruction-tuned language models besides best performances. 
Instruction-tuning has shown promise in improving zero-shot performance of LLMs, especially for comparatively small models \cite{alpaca,longpre2023flan,sanh2021multitask}.
Indeed, careful instruction-tuning of ``smaller'' models like T5-XL (3B parameters) can result in zero-shot performance that is competitive with---or even better than---much larger models \cite{longpre2023flan}.
Realizing such functionality with modestly sized models is an important and timely research goal, and instruction-tuning has emerged as a promising means to this end.

The promise of such models lies in their ability to perform novel tasks specified via natural language instructions.
It is therefore important that models are robust to variations in instruction phrasings: Semantically equiavelent instructions should yield comparable results.
However, we find that instruction-tuned LLMs are (sometimes very) sensitive to changes in phrasings, with novel instructions (i.e., instructions unobserved in training) leading to substantially degraded performance.

%For a specific task, a \textbf{robust} model, after instruction-tuning, ought to perform equally well when given two semantically equivalent instructions
\end{comment}
\begin{comment}
\begin{figure}%[h]
  \centering
  \includegraphics[scale=0.45]{images/data_cropped.pdf}
  \caption{To evaluate the generalization capabilities of LLMs on unobserved instruction and tasks, we collect additional instruction templates in two ways: (1) Automatically paraphrasing from the original Flan collection, and (2) Enlisting NLP researchers to manually compose instructions for tasks. In total, we collected 1206 instruction templates which we use to evaluate the robustness of instruction-tuned models.}
  %In total, we collected a set of 1206 instruction templates to compare with the performance of instructions observed during training }
  \label{fig:data}
\end{figure}
\end{comment}

\begin{comment}

\begin{figure}[h]
  \centering
  \includegraphics[scale=0.5]{images/robustness_cropped.pdf}
  \caption{In this example, the observed instruction is the \textit{Task 1286} from NIV2 \cite{wang2022benchmarking} using \textit{Template 6}, while the unobserved instruction shown was composed by an NLP practitioner specifically for MMLU \cite{hendrycks2020measuring}. The model fails in this latter case, despite the instruction being valid.}
  \label{fig:my_label}
\end{figure}
\end{comment}

\section{Related Work}
\label{section:related-work}

\paragraph{Multitask learning and instruction-tuning}

%Before the instruction era, previous works of multi-task fine-tuning focused on bettering the model's NLU ability by unifying several downstream tasks into one and fine-tuning over the unified corpus 
%Prior to the introduction of \emph{instruction-tuning}, 
Training a single text-to-text model capable of providing responses to arbitrary queries has been an aspiration in NLP for at least half a decade. 
%pre-dates the current wave of LLMs and modern prompting and instructing strategies. 
%Prior to modern prompting and instructing strategies, 
For example, prior to modern prompting and instructing strategies, there were efforts to unify disparate tasks by reframing them as instances of general \emph{question answering} \cite{mccann2018natural, khashabi2020unifiedqa, keskar2019unifying}. 
%After that, many contemporary works carry out meta-dataset resources for multi-task tuning with instructions to unlock the hidden knowledge learned through large-scale unsupervised learning, the representative works are: Flan 2021 collection \cite{wei2021finetuned}, Natural Instructions \cite{mishra2021cross} and T0 \cite{sanh2021multitask}. 
More recent efforts have focussed on compiling and fine-tuning LLMs on corpora comprising diverse tasks with associated natural language instructions \cite{wei2021finetuned,mishra2021cross,sanh2021multitask}; we refer to this strategy as instruction-tuning. 
One example of this is {\tt Super-NaturalInstructions} \cite{wang2022benchmarking}, which compiles over 1600 tasks and enriches these with both instructions and negative examples. %to enrich the instruction collection. On top of that, t
Similarly, the recently released OPT-IML Bench \cite{iyer2022opt} comprises 2000 NLP tasks. %covering a range of categories.
The Flan 2022 task collection \cite{longpre2023flan} additionally features \emph{Chain-of-Thought} (CoT) style ``reasoning'' chains in instruction templates; the authors show that including these (as well as zero-shot examples and ``input inversions'') during instruction fine-tuning yields improvements on held-out tasks. 
%he new Flan collection further incorporates Chain-of-Thought (CoT) reasoning into the instruction templates and shows that it benefits instruction fine-tuning. 

These meta-resources---collections of instructions, tasks, and samples---have facilitated the training of instruction-tuned model families such as Flan-T5, Flan-PaLM \cite{chung2022scaling}, and OPT-IML \cite{iyer2022opt}.\footnote{Somewhat confusingly, in the case of FLAN and OPT, the corpora (i.e., benchmarks comprising tasks and instructions) and LLMs fine-tuned using them are both referred to with the associated acronym as prefix: For instance, Flan-T5 denotes a T5 \cite{raffel2020exploring} variant fine-tuned with the Flan collection.}
Results have been encouraging; fine-tuning LLMs to follow instructions provides clear and consistent gains across models, and, perhaps most exciting, enables relatively ``small'' ($\sim$10B) LLMs to achieve near SOTA performance comparable to massive ($\sim$175B) models \cite{alpaca}. 
This has motivated interest in characterizing how instructions help models, and developing techniques to further improve instruction-tuning; we review recent efforts related to these two research threads below. 

%Subsequently, with all these resources available many instruction-tuned models like Flan-T5, Flan-PaLM \cite{chung2022scaling}, and OPT-IML \cite{iyer2022opt} are trained and achieved state-of-the-art performance on many downstream tasks.
\paragraph{Evaluating prompting and instruction capabilities}
%Inherited from the prompt learning paradigm, many works aim to evaluate the hidden mechanism of instruction tuning and its potential flaws. 
Instructions may be seen as a special sort of model prompting, which a few recent efforts have critically evaluated. 
For example, Webson and Pavlick ask whether models meaningfully ``understand'' prompts \cite{webson2021prompt}, finding that they largely do not: %looks at the choices of prompt for inference over instruction fine-tuned model and found that the 
Performance is often unaffected when irrelevant and misleading prompts are provided. 
In follow up work, Jang \emph{et al.}
\cite{jang2023can} evaluates performance on negated prompts, observing an ``inverse-scaling'' phenomenon in which larger models perform worse in this case.
%studies the negated prompt and found that it inversely scaled with the size of the model. 

Other work has attempted to characterize how and when \emph{in-context learning} (ICL)---i.e., including a few examples in prompts---works \cite{min2022rethinking,wang2023large,dai2022can,akyurek2022learning,yu2022alert}. 
ICL is a form of prompting orthogonal to the present effort, as we are primarily interested in the zero-shot adaptability of instruction-tuned LLMs.

%With respect to instruction tuned models specifically, Yu \emph{et al.} \cite{yu2022alert} show that instruction tuning enhances zero-shot performance of the model at the cost of harming its ability to follow various instructions. 
%\cite{min2022rethinking} examine the performance of instruction-tuned models on classification tasks and discover that altering the label space in the instruction has a marginal impact on the performance.

In work contemporaneous to ours, Gu \emph{et al.} \cite{gu2023robustness} investigated how robust instruction-tuned models are to instruction perturbations (e.g., dropping words) and paraphrasings. 
They found that models are relatively robust when given examples (i.e., in few-shot settings), but quite sensitive when used zero-shot; this is qualitatively in line with our findings.
Our work differs in important way from this coincident research: (1) We provide a much more comprehensive analysis of robustness; Gu \emph{et al.} considered \emph{only} T5 instruction-tuned on a single instruction dataset, whereas we evaluate three LLMs (and different sizes of each) using five instruction tuning datasets, and we evaluate using over 80 test tasks in all (Gu \emph{et al.} considered only 12). (2) We propose and evaluate a new approach to \emph{improving} the robustness of instruction-tuned models; Gu \emph{et al.} offered no mechanism to improve robustness. 


\paragraph{Improving instruction-tuning}
%Most of the works aim to improve the performance of instruction-tuned models are focusing on two threads. 
Past work has also sought to improve instruction-tuning in various ways.
One means to do so is to instruction tune based on human feedback \cite{ouyang2022training, glaese2022improving, bai2022training, nakano2021webgpt, zhang2023wisdom}. 
This tends to improve open-ended model responses but degrade performance on downstream tasks. %improves the model's open-ended task performance at the cost of NLP tasks performance degradation. 
Another strategy is to leverage existing resources to automatically generate instruction-tuning datasets at scale. 
For example, Wang \emph{et al.} \cite{wang2022self} use LLMs to generate instructions, inputs, and outputs and use these to improve their own instruction-following capabilities. 
%improves the instruction-following capabilities of LLMs by bootstrapping their own generations to train the model.
In a similarly meta vein, Zhou and colleagues \cite{zhou2022large}  propose using LLMs to engineer prompts. 
%regards instructions as program to perform text-to-structure generation with LLM. 
Finally, Ye \emph{et al.} \cite{ye2022guess} propose ``flipping'' the standard task by tasking LLMs with generating \emph{instructions}, given an input and label. 
%trains the LM to produce instructions given the input and labels. 
%\input{sections/03_instruction_robustness.tex}
\section{Instruction Datasets}

\subsection{Evaluation Benchmarks}

%We perform the evaluation of instruction-tuned models 
We evaluate a set of instruction-tuned models on two large benchmarks: \textsc{MMLU} \cite{hendrycks2020measuring} and \textsc{Big-Bench} \cite{srivastava2022beyond}. \textsc{MMLU} is a multiple-choice question-answering benchmark comprising 57 tasks that require expert knowledge.
\textsc{Big-Bench} is a collaboratively built benchmark containing 204 diverse tasks from various domains; here %we use the 18 task 
consider the \textsc{Big-Bench Lite} subset, and we include only QA, multi-class, and binary classification tasks, yielding 18 tasks from in all. %in \textsc{Big-Bench}.

%More specifically, We conduct our experiment over all 57 tasks on MMLU. An 18 tasks subset of \textsc{Big-Bench Lite} consists of all QA, multi-class, and binary classification tasks.

\subsection{Collecting New Instructions from NLP Researchers}
\label{section:new-instructions}

We aim to evaluate instruction-tuned models when they are provided instructions which are semantically equivalent to, but superficially different from, those with which they were trained.
To this end, we enlist NLP researchers (graduate students) to compose novel instructions for the tasks considered; these particular instruction phrasings were therefore \emph{unobserved} during instruction fine-tuning. 

%For each instruction-tuned language model we  we collect a set of instructions unobserved to the model by expert annotation, and we collect a set of instructions observed during training from the original instruction-tuning collection.

% TODO: to have some small tables here to show the stats

%\subsubsection{Unobserved Instruction}

%We perform large-scale crowd-sourcing from 
More specifically, we recruited 36 NLP graduate students working in NLP.
All had at least some experience with instruction-tuned models and the downstream tasks included in the evaluation benchmarks. 
For each of the 18 tasks in \textsc{BBL} and all tasks in \textsc{MMLU}, we asked 12 graduate students to write one (distinct) instruction they would use for zero-shot inference with an instruction-tuned model. 
%To ensure fairness, the information on models is omitted to avoid having priors to fit the pattern of the specific model. 
%The detailed instruction collection process can be seen in Appendix A.
We provide details on this instruction collection process in Appendix A. 
We will release all 319 instructions acquired for this work to ensure the reproducibility of this work and to facilitate further research on instruction-tuned model robustness. 
% We treat 57 tasks of MMLU as a whole (general QA template). Don't know how to make the word clearer

%\subsubsection{Observed Instruction}


% could be generally applied to one of ``multiple-choice QA'', ``binary label classification'', and ``m
%``multi-label classification" tasks. 

% TODO: to have some small tables here to show the stats






\section{Evaluating the Robustness of Instruction-tuned LLMs}

\subsection{Models and Data}

We conduct experiments with model variants trained over three instruction collections (these provide \emph{observed} task instructions): P3 \cite{sanh2021multitask}, Flan-2022 \cite{chung2022scaling}, and Alpaca \cite{alpaca}.
%For each model and associated published instruction collection, we manually perused entire collection and pick the 
To facilitate our analyses, we manually identified all instructions that correspond to (a) multiple-choice question answering (QA), (b) binary classification (BC), or tasks that demand ``yes'' or ``no'' responses, and (c) multi-class classification (MC), which requires classifying inputs into a finite set of categories.

To evaluate model robustness with respect to instruction phrasings we use two benchmarks: \textsc{MMLU} \cite{hendrycks2020measuring} and \textsc{Big-Bench Lite} (\textsc{BBL}) \cite{srivastava2022beyond} along with the acquired set of novel instructions described in Section \ref{section:new-instructions}.
%We evaluate all tasks with multi-choice grading with logit scores to keep a fair comparison. 
We include all 57 tasks from \textsc{MMLU}, and 14 of 24 tasks from \textsc{BBL}.
From the latter we exclude two tasks that rely on generation metrics, four that use exact-match, and four that contain tokens unrecognized by the T5 and/or LLaMa tokenizer (e.g., inputs are emojis in one task).

%2 tasks with NLG metric, 4 tasks with exact-match metric, and 4 tasks containing T5/LLaMa unrecognized tokens were removed from evaluation.
%For unobserved instruction, we follow the procedure in section 3.2 to collect task-specific instructions for each task that we are evaluating. 
%For observed instruction, we classify all tasks into three categories: 
%We group observed instructions under three categories corresponding to task types (which imply particular output formats): (1) Multiple-choice QA, i.e., tasks that entail selecting an answer to a question from a finite set of options;
%that present themselves as questions with choices. 
%(2) Multi-class classification, which requires classifying inputs into a finite set of categories; and (3) Binary classification, or tasks that demand ``yes'' or ``no'' responses. 

\begin{table}[h]
  \small
  \centering
  \begin{tabular}{c l}
    \toprule
    \multirow{3}{*}{\textsc{QA}} & In this task, you are given a multiple-choice question and you have to pick the                                                                                                                              \\
                                 & correct option. Answer with option indexes (i.e., "A", "B", "C", and "D").                                                                                                                                   \\
                                 & Q: \textcolor{ForestGreen}{\{question\}} A. \textcolor{MidnightBlue}{\{choiceA\}} B. \textcolor{MidnightBlue}{\{choiceB\}} C. \textcolor{MidnightBlue}{\{choiceC\}} D. \textcolor{MidnightBlue}{\{choiceD\}} \\
    \midrule
    \multirow{1}{*}{\textsc{MC}} & Pick one category for the following text. The options are - \textcolor{MidnightBlue}{\{options\}} \textcolor{ForestGreen}{\{text\}}                                                                          % \\
    %& \textcolor{ForestGreen}{\{text\}} \\
    \\
    \midrule
    \multirow{2}{*}{\textsc{BC}} & \textcolor{ForestGreen}{\{paragraph\}} Choose your answer: According to the above paragraph, the                                                                                                             \\
                                 & question "\textcolor{ForestGreen}{\{question\}}" is "\textcolor{MidnightBlue}{\{response\}}"?                                                                                                                \\
    \bottomrule
  \end{tabular}
  \caption{Examples of observed instructions we collected for three general types of tasks.}
  \label{table:instruction-examples}
\end{table}

We use the same instructions for all tasks in the same category, taken from the published instruction tuning datasets associated with each model.
These instructions are general, e.g., in the case of classification they request that the model consider an example with respect to categorization criteria and label space provided by the instance, and select an appropriate category (examples in Table \ref{table:instruction-examples}).
One can ``mix-and-match'' such instructions so long as they are appropriate for the task type.


%this means there will be cases where we use an ``incorrect'' instruction for a particular dataset, i.e., instructing the model to select an answer on a basis that is in fact irrelevant to the task being considered (but such that the elicited output format will be correct).
%This may degrade the performance of ``observed'' instructions, as compared to results which might be obtained if one manually aligned instructions to datasets within benchmarks. 
%We made this analysis decision to avoid biasing results by inflating the performance of ``observed'' instructions; we are interested in how robust instruction-tuned LLMs are



%3) Binary-classification: tasks that require the answer 'yes' or 'no'. 
%To avoid subjective bias, we use the same instructions for all tasks in the same category. They are collected from the published instruction tuning dataset for each model.

\begin{table}%[h]
  \centering
  \small
  \begin{tabular}[t]{l l c c c}
    \multicolumn{5}{c}{\textsc{Observed Instructions}}         \\
    \toprule
    %& \textsc{MMLU} & \multicolumn{3}{c}{\textsc{BBL}} \\
    %& \multicolumn{3}{c}{\textsc{MMLU}} & \multicolumn{1}{c}{\textsc{BBL}} \\
    \emph{Instruction Type} & \multicolumn{2}{c}{QA} & MC & BC \\
    Flan                    & \multicolumn{2}{c}{50} & 35 & 18 \\
    Alpaca                  & \multicolumn{2}{c}{20} & 20 & 11 \\
    P3                      & \multicolumn{2}{c}{13} & 8  & 7  \\
  \end{tabular}
  \quad
  \begin{tabular}[t]{l l|c|c|c}
    \multicolumn{5}{c}{\textsc{Unobserved Instructions}}                     \\
    \toprule
    Number of tasks       & \multicolumn{2}{c}{1}  & \multicolumn{2}{c}{14}  \\
    Instructions per task & \multicolumn{2}{c}{20} & \multicolumn{2}{c}{10}  \\
    \hline
    Total instructions    & \multicolumn{2}{c}{20} & \multicolumn{2}{c}{140} \\
  \end{tabular}

  \caption{Counts of instruction phrasings (unobserved and observed) we use for evaluations.}
  \label{tab:data_stat}
\end{table}


\begin{comment}
\begin{table}[h]
  \centering
  \begin{tabular}{l l|c|c|c}
    \multicolumn{5}{c}{\textsc{Unobserved Instructions}}                                         \\
    \toprule
    %\hline
                          & \multicolumn{2}{c}{\textsc{MMLU}} & \multicolumn{2}{c}{\textsc{BBL}} \\
    Number of tasks       & \multicolumn{2}{c}{1}             & \multicolumn{2}{c}{12}           \\
    Instructions per task & \multicolumn{2}{c}{20}            & \multicolumn{2}{c}{20}           \\
    \hline
    Total instructions    & \multicolumn{2}{c}{20}            & \multicolumn{2}{c}{120}          \\
  \end{tabular}
  \begin{tabular}{c c|c|c|c}
    \multicolumn{5}{c}{\textsc{Observed Instructions}}                                  \\
    \toprule
    %\hline
                     & \textsc{MMLU}            & \multicolumn{3}{c}{\textsc{BBL}}      \\
    \hline
    Instruction Type & \multicolumn{2}{|c|}{QA} & MC                               & BC \\
    Flan             & \multicolumn{2}{|c|}{50} & 20                               & 18 \\
    Alpaca           & \multicolumn{2}{|c|}{20} & 20                               & 11 \\
    P3               & \multicolumn{2}{|c|}{13} & 8                                & 7  \\
  \end{tabular}
  \caption{Counts of instruction phrasings (unobserved and observed) we use for evaluations.}%The numbers of observed and unobserved instructions that we used for each model that we evaluate}
  \label{tab:data_stat}
\end{table}
\end{comment}

%Hence,
\begin{comment}
Formally, for each dataset/task $\mathcal{D}^i$, we have a set of unobserved instructions $\mathcal{I}_{\text{unb}}^{i}$ and a set of observed instructions $\mathcal{I}_{\text{obs}}^{i}$.
We perform inference using these instructions over each dataset and calculate aggregate statistics (average accuracies and standard deviations) for performance achieved using $\mathcal{I}_{\text{unb}}^{i}$ and $\mathcal{I}_{\text{obs}}^{i}$.
\end{comment}
%For each individual instruction, we apply it to the dataset and run the inference. 
%We compute and report the average accuracy and standard deviation of all the instructions in the set as the final result for that dataset. 

\subsection{Results}
\label{section:main-analysis-results}

We present the main aggregated analysis results in Figure \ref{fig:main-results} and Table \ref{tab:main_result}.
The take-away here is that using instructions unobserved in training---but manually composed for the task at hand and so semantically appropriate---leads to considerable degradation in performance: On average, unobserved instructions reduce accuracy by over five points across models considered.
Table \ref{tab:main_result} reports results disaggregated by task type; we observe that classification tasks are most harmed by use of novel instructions.
We provide additional, more granular (dataset-level) results in the Appendix.

\begin{figure}[htbp]
  \centering
  \begin{subfigure}{0.475\textwidth}
    %\includegraphics[width=\textwidth]{images/updated_plot.pdf}
    \caption{Average zero-shot performance over all tasks when using observed and unobserved instructions.}
    \label{fig:main-results-main_results}
  \end{subfigure}
  \hfill
  \begin{subfigure}{0.475\textwidth}
    %\includegraphics[width=\textwidth]{images/scaling_plot_larger.pdf}
    \caption{Performances of Flan-T5 using observed and unobserved instructions as a function of model size.}
    \label{fig:main_scaling_reesults}
  \end{subfigure}
  \caption{Using novel but valid instructions at test time (phrasings unobserved in training) consistently degrades the performance of instruction-tuned LLMs (a). Scale does not necessarily fix this (b).}
  \label{fig:main-results}
\end{figure}

\begin{comment}
\begin{figure}
  \centering
  \includegraphics[width=14.4cm]{images/main_results_v4.pdf}
  \caption{Average performance across tasks between observed and unobserved instructions.}
  \label{fig:main_results}
\end{figure}

\begin{figure}
  \centering
  \includegraphics[width=14.4cm]{images/main_scaling_results_v1.pdf}
  \caption{}
  \label{fig:main_scaling_results}
\end{figure}
\end{comment}


\begin{table}[h]
  \small
  \centering
  \begin{tabular}{l c c c c c}
    \toprule
    \multirow{2}{*}{\textbf{Model}}               & \textsc{MMLU}                           & \textsc{BBL-QA}                         & \textsc{BBL-BC}                    & \textsc{BBL-MC}                    & \textbf{Overall}                  \\ [0.5ex]
                                                  & Avg. \ \ Std.                           & Avg. \ \ Std.                           & Avg. \ \ Std.                      & Avg. \ \ Std.                      & Avg. \ \ Std.                     \\
    \hline
    \rule{0pt}{12pt} Flan-T5-3B                   &                                         &                                         &                                    &                                                                        \\
    \hspace{0.25cm} \textsc{Observed}             & $\textbf{48.1} \ \ (\pm 0.3)$           & $\textbf{59.0} \ \ (\pm 2.1)$           & $\textbf{66.5} \ \ (\pm 3.8)$      & $\textbf{55.6} \ \ (\pm 0.7)$      & $\textbf{57.3} \ \ (\pm 1.7)$     \\
    \hspace{0.25cm} \textsc{Unobserved}           & $47.5 \ \ (\pm 0.9)$                    & $56.0 \ \ (\pm 7.3)$                    & $61.1 \ \ (\pm 6.9)$               & $52.1 \ \ (\pm 5.4)$               & $54.2 \ \ (\pm 5.1)$              \\
    \hspace{0.25cm} \textbf{Performance $\Delta$} & \textcolor{red}{$\downarrow 0.6$}       & \textcolor{red}{$\downarrow 3.0$}       & \textcolor{red}{$\downarrow 5.5$}  & \textcolor{red}{$\downarrow 3.5$}  & \textcolor{red}{$\downarrow 3.1$} \\
    %\hline 
    \rule{0pt}{12pt} Alpaca-7B                    &                                         &                                         &                                    &                                                                        \\
    \hspace{0.25cm} \textsc{Observed}             & $\textbf{41.9} \ \ (\pm 0.6)$           & $\textbf{48.6} \ \ (\pm 2.8)$           & $\textbf{53.8} \ \ (\pm 3.4)$      & $\textbf{32.1} \ \ (\pm 2.2)$      & $\textbf{44.1} \ \ (\pm 2.3)$     \\
    \hspace{0.25cm} \textsc{Unobserved}           & $39.7 \ \ (\pm 2.2)$                    & $45.3 \ \ (\pm 6.5)$                    & $52.4 \ \ (\pm 6.5)$               & $16.4 \ \ (\pm 3.5)$               & $38.5 \ \ (\pm 4.7)$              \\
    \hspace{0.25cm} \textbf{Performance $\Delta$} & \textcolor{red}{$\downarrow 2.2$}       & \textcolor{red}{$\downarrow 3.3$}       & \textcolor{red}{$\downarrow 1.4$}  & \textcolor{red}{$\downarrow 15.7$} & \textcolor{red}{$\downarrow 5.6$} \\
    %\hline 
    \rule{0pt}{12pt} T0++ 11B                     &                                         &                                         &                                    &                                                                        \\
    \hspace{0.25cm} \textsc{Observed}             & $48.3 \ \ (\pm 0.9)$                    & $54.1 \ \ (\pm 4.1)$                    & $\textbf{66.1} \ \ (\pm 2.1)$      & $\textbf{42.0} \ \ (\pm 2.1)$      & $\textbf{52.6} \ \ (\pm 2.3)$     \\
    \hspace{0.25cm} \textsc{Unobserved}           & $\textbf{48.5} \ \ (\pm 0.9)$           & $\textbf{54.7} \ \ (\pm 3.7)$           & $54.7 \ \ (\pm 4.3)$               & $41.4 \ \ (\pm 2.4)$               & $49.8 \ \ (\pm 2.8)$              \\
    \hspace{0.25cm} \textbf{Performance $\Delta$} & \textcolor{ForestGreen}{$\uparrow 0.2$} & \textcolor{ForestGreen}{$\uparrow 0.7$} & \textcolor{red}{$\downarrow 11.4$} & \textcolor{red}{$\downarrow 0.6$}  & \textcolor{red}{$\downarrow 2.8$} \\
    %\hline 
    \rule{0pt}{12pt} Flan-T5-11B                  &                                         &                                         &                                    &                                                                        \\
    \hspace{0.25cm} \textsc{Observed}             & $\textbf{53.2} \ \ (\pm0.2)$            & $\textbf{67.9} \ \ (\pm1.8)$            & $\textbf{65.6} \ \ (\pm6.0)$       & $\textbf{58.7} \ \ (\pm0.5)$       & $\textbf{61.4} \ \ (\pm2.1)$      \\
    \hspace{0.25cm} \textsc{Unobserved}           & $52.7 \ \ (\pm0.8)$                     & $64.6 \ \ (\pm8.5)$                     & $63.6 \ \ (\pm6.1)$                & $55.9 \ \ (\pm5.5)$                & $59.2 \ \ (\pm5.2)$               \\
    \hspace{0.25cm} \textbf{Performance $\Delta$} & \textcolor{red}{$\downarrow 0.5$}       & \textcolor{red}{$\downarrow 3.4$}       & \textcolor{red}{$\downarrow 2.0$}  & \textcolor{red}{$\downarrow 2.8$}  & \textcolor{red}{$\downarrow 2.2$} \\
    %\hline
    \rule{0pt}{12pt} Alpaca-13B                   &                                         &                                         &                                    &                                                                        \\
    \hspace{0.25cm} \textsc{Observed}             & $\textbf{47.8} \ \ (\pm 0.5)$           & $\textbf{53.9} \ \ (\pm 2.2)$           & $\textbf{57.9} \ \ (\pm 4.8)$      & $\textbf{36.7} \ \ (\pm 1.8)$      & $\textbf{49.1} \ \ (\pm 2.3)$     \\
    \hspace{0.25cm} \textsc{Unobserved}           & $47.0 \ \ (\pm 0.8)$                    & $51.7 \ \ (\pm 5.7)$                    & $54.1 \ \ (\pm 5.6)$               & $22.7 \ \ (\pm 7.5)$               & $43.9 \ \ (\pm 14.0)$             \\
    \hspace{0.25cm} \textbf{Performance $\Delta$} & \textcolor{red}{$\downarrow 0.9$}       & \textcolor{red}{$\downarrow 2.2$}       & \textcolor{red}{$\downarrow 3.8$}  & \textcolor{red}{$\downarrow 14.0$} & \textcolor{red}{$\downarrow 5.2$} \\
    \bottomrule
  \end{tabular}
  \vspace{0.25cm}
  \caption{Results using observed and unobserved instructions across benchmark tasks (grouped by type). Performance degrades---sometimes by 10+ points---when one uses (\textsc{unobserved}) instructions, suggesting that instruction-tuned models are not particularly robust. BC, MC, and QA stand for binary classification, multi-class classification, and question answering, respectively.} %The overall performance of instruction-tuned models with observed and unobserved data. \textsc{MC} stands for multiple classification, and \textsc{BC} stands for binary classification.}
  \label{tab:main_result}
\end{table}

\subsection{A Closer Look at Instruction Robustness}
\label{section:closer-look}

Above we used general instructions requesting the model to perform tasks (Table \ref{table:instruction-examples}).
%to avoid data contamination due to researcher bias. 
Here we delve further into the performance degradation observed when using novel instructions. % that results from using unobserved instructions at inference time, 
We report a curious result highlighting the degree to which models rely on having previously observed instructions: Incorrect but observed instructions outperform appropriate but unobserved instructions (Figure \ref{fig:adversarial}).
%degree to which these models depend on having observed an instruction given at inference time. 

We come to this observation by evaluating the performance of Flan-T5-XXL (11B) using six instruction types over seven datasets from \textsc{Big-Bench}. %, following the same protocol as in \ref{section:main-analysis-results}, 
%we evaluate the performance of Flan-T5-XXL (11B) with six different instruction types. %under six different settings. 
In particular, this includes (variants of) two instructions \emph{observed} in training:
\textbf{Closest} is the instruction from the most similar task in the instruction-tuning set; \textbf{Incorrect} is an observed instruction for a \emph{completely different} and inappropriate task (but which has the same desired output format, e.g., classification)---intuitively these should not yield the desired behavior; \textbf{Negated} is the same as \textbf{closest}, but we negate the instruction to indicate that it should \emph{not} perform the task.
%to reverse the instruction.


For \emph{unobserved} instructions, we consider:
\textbf{Task designer}, the instruction (task prefix) provided by the author of the task in \textsc{Big-Bench}, and;
\textbf{Newly collected}, or the novel instructions collected from NLP graduate students, described above.
As a control for reference, we also consider \textbf{Nonsensical}, which is a random ``instruction'' completely irrelevant to any task.


\begin{comment}

\begin{figure}
  \centering
  \includegraphics[width=140mm]{images/adversarial_w_examples_cropped.pdf}
  \caption{Caption}
  \label{fig:adversarial}
\end{figure}
\end{comment}

\begin{figure}
  \centering
  %\includegraphics[scale=0.335]{images/adversarial.pdf}
  \caption{{\bf \emph{Incorrect} but observed instructions perform better on average than \emph{correct} but unobserved instructions}. We report averages over benchmarks, but show example instructions on the right for a specific, illustrative task. We provide all instructions in the Appendix.}
  \vspace{-0.5em}
  \label{fig:adversarial}
\end{figure}

Figure \ref{fig:adversarial} reports average results for these variants.
Consistent with our findings, using instructions unobserved in training degrades performance.
Strikingly, here we also find that using an \emph{inappropriate but observed} instruction outperforms using \emph{appropriate but unobserved} instructions.
This indicates that instruction-tuned models---or at least modestly sized ones we have evaluated here---may in some way overrely on having observed instructions in training, and do not generalize to new instructions and phrasings as we might hope. We provide all the instructions and results in the Appendix.
%generalize somewhat poorly to new instructions, compared to their performance when used with instructions seen in training.
%That observed but incorrect instructions fare better than unobserved but correct instructions illustrates this point. 


%The result shows that in most of cases, the observed instructions outperform the unobserved ones by a large margin. More surprisingly, the incorrect, observed instructions - being semantically distracting to the actual task - have comparable performance with the unobserved instruction, which is semantically accurate but not trained before.


% Specifically, we ... % TODO
%Here, to provide a closer look at the gap between the performance of observed and unobserved instructions, we conduct a more specific experiment that tests the  

\begin{comment}
\begin{table}[h]
  \centering
  \begin{tabular}{l c}
    \toprule
    Settings             & \textbf{Avg. Acc.} \\ [0.5ex]
    \hline
    Task Designer        & 45.2               \\
    Unobserved           & 44.1               \\
    \hline
    Observed - Correct   & \textbf{50.5}      \\
    Observed - Incorrect & 47.2               \\
    Observed - Negated   & 43.7               \\
    \hline
    Random Text          & 35.9               \\
    \bottomrule
  \end{tabular}
  \caption{Results on 7 Datasets from \textsc{Big-Bench} with different instruction settings.}
  \label{tab:my_label}
\end{table}
\end{comment}

\subsection{Scaling}

%Due to the limited accessibility of LLMs instruction-tuning collections, we are not able to evaluate the instruction robustness of 
Does instruction robustness begin to emerge as a function of scale?
To attempt to answer this, we repeated all experiments from Table \ref{tab:main_result} with Flan-T5 model sizes ranging from small (80M parameters) to XXL (11B).
We observe in Figure \ref{fig:main_scaling_reesults} that the disparity between results achieved with observed versus unobserved instructions \textbf{does not} seem to decrease with model scale, at least up to this point.
That said, massive models (175B+) may offer greater robustness.
However, we reiterate that much of the excitement about instruction tuning is the possibility that this technique appears to allow much smaller models to achieve results competitive with massive alternatives.
%the scalability of the issue that we have discovered, we took Flan-T5, the instruction-tuned model with the largest size variation, by repeating all the experiments in \ref{tab:main_result} from Flan-T5-Small (80M) to Flan-T5-XXL (11B). The performance gap on both \textsc{MMLU} and \textsc{BBL} is not decreased as the model scales exponentially.


\subsection{Robustness with Semantic Distance}
\label{section:mmlu_variance}
%Results Not So Variable on MMLU?}

One observation in \ref{section:main-analysis-results} is that performance on \textsc{MMLU} is less affected by using unobserved instructions.
%seems to suffer less from the O.O.D. instructions. 
\textsc{MMLU} is a benchmark with 57 QA tasks about different knowledge domains; these tasks all share a similar form of input-output (question, four choices $\rightarrow$ answer).
During instruction collection, we treated all tasks in \textsc{MMLU} as a general QA task and asked NLP researchers to write general QA instructions.
%to give instructions that can apply to all the questions. Hence, these "meta" 
As a result, we hypothesize that these instructions are comparatively similar to the observed instructions, and this in turn explains the relative robustness in this case.
% that we collected for evaluation. Therefore, the degradation of the performance is relatively light.

We empirically verify this in Figure \ref{fig:embeddings} and Table \ref{table:distances}. For each instance (instruction plus example), we extract the representation at the penultimate layer for the first decoded token. %To visualize the distribution of instances with observed and unobserved instructions, w
We use tSNE \cite{van2008visualizing} to visualize these representations of observed and unobserved instructions over instances in \textsc{MMLU} and \textsc{BBL}.
Figure \ref{fig:embeddings} shows that in the case of \textsc{MMLU} the unobserved instructions we collected are quite similar to the observed, while there is a greater separation between unobserved and observed instructions in \textsc{BBL}.
We also provide a numerical measurement of this phenomonen in Table \ref{table:distances}.
We report the average $\ell$2 distance between representations of unobserved instructions and those of their nearest observed counterparts.
We see that \textsc{MMLU} unobserved instructions are, on average, closer to the nearest observed instruction; this correlates with the lower observed performance drop.
These findings are in line with the hypothesis that the unobserved instructions for \textsc{MMLU} are more similar to the observed instructions for this dataset, and this likely explains the apparent robustness in this case. %We report analogous results for all datasets in the Appendix.
%we provide a numerical measurement of what the plot is indicating. For each unobserved instruction, we match it with the closest observed instruction by the average L2 distance across all the examples that we sampled. We compute the average distances and accuracy degradation with all the observed and unobserved instructions pairs.
%, we show that the instructions we collected for MMLU and BBL have an astonishing difference. In MMLU, the instances with observed instruction and unobserved instruction are less distinguishable, whereas in BBL, the two clusters are clearly separated in the latent space.

%To verify our hypothesis, propose a method to measure the distributional difference between observed and unobserved instructions. 

\begin{figure}[h]
  \centering
  %\includegraphics[scale=0.255]{images/exp_emb_plot_300_cropped.pdf}
  \caption{tSNE plots of representations for the first decoded tokens of 300 randomly sampled examples from \textsc{MMLU} and \textsc{BBL} with Flan-T5 (XXL). Embeddings of observed and unobserved instructions for \textsc{MMLU} are similar, while for \textsc{BBL} they are quite different. This result holds across most but not all models considered: See the \ref{section:embeddings} for visualizations over all models.}%The pattern of the tSNE plot is not an absolute metric and varies from model to model based on the embedding size. However, most of the examples we visualize show similar patterns as demonstrated, and we report all of the visualizations in the Appendix.}
  \label{fig:embeddings}
\end{figure}


\begin{comment}
\begin{table}[h]
  \small
  \centering
  \begin{tabular}{l l l}
    \toprule
    \textbf{Dataset}                 & \textbf{Avg. $\ell$2} ($\ell$2) & \textbf{Avg. $\Delta$ Accuracy} (\%) \\
    \midrule
    \textsc{MMLU}                    & \textbf{19.8}                   & -\textbf{1.5}\%                      \\
    \midrule
    \textsc{Novel Concepts}          & 22.0                            & -3.1\%                               \\
    \textsc{StrangeQA}               & 55.3                            & -5.5\%                               \\
    \textsc{Language Identification} & 59.0                            & -11.3\%                              \\
    \bottomrule
  \end{tabular}
  \caption{Average distances and accuracy degradations (as \%) on three datasets in \textsc{BBL}.}
  \label{table:distances}
\end{table}
\end{comment}

%The result shows that the accuracy degradation for using unobserved instructions is highly correlated with the similarity between the instructions used and the instruction trained. 

We plot mean performance degradation (as \%) as a function of average similarity between the similarity of the first decoded tokens (following \emph{unobserved} instructions) and the same for the \emph{most similar} \emph{observed} instruction.
The negative slope implies the intuitive relationship: Instructions that are dissimilar (in terms of model representations) tend to result in poorer performance.  However, the relationship is relatively weak, yielding an intercept estimate of -0.8 and a slope of -0.2 ($p=$0.08).


\begin{figure}[h]
  \begin{floatrow}
    \floatbox{figure}[.5\textwidth][\FBheight][t]{
      \centering
      %\includegraphics[scale=0.45]{images/reg_cropped.pdf}
    }
    {
      \caption{Plots of average degradations in performance versus the semantic distance while using unobserved instructions.}
      \label{fig:perf-dist-reg}
    }
    % \label{fig:perf-dist-reg}
    \floatbox{table}[.5\textwidth][\FBheight][t]{
      \vspace{-30pt}
      \begin{tabular}{l l l}
        \toprule
        \textbf{Dataset} & \textbf{Avg.} $\Delta\ell$2 & \textbf{Avg. $\Delta$ Acc.} \\
        \midrule
        \textsc{MMLU}    & \textbf{19.8}               & -\textbf{0.5}               \\
        \midrule
        \textsc{BBL-QA}  & 37.9                        & -3.4                        \\
        \textsc{BBL-BC}  & 25.3                        & -2.0                        \\
        \textsc{BBL-MC}  & 26.1                        & -2.8                        \\
        \bottomrule
      \end{tabular}
    }
    {
      \caption{Average degradations in performance for four categories. It could be seen that \textsc{MMLU} has minimal average distance, which indicates a smaller distribution shift, and hence leads to the smallest degradation}
      \label{table:distances}
    }
  \end{floatrow}
\end{figure}

\begin{comment}
\caption{Average degradations in performance observed when using instructions unobserved training as a function of the similarity between (a) the representation induced by the model for a given instruction, and, (b) the same for the \emph{nearest} observed instruction. This is for Flan-T5 (XXL). We use representations for the first token following the instruction, extracted from the penultimate layer in the network.}
\end{comment}

\vspace{-0.5em}
\subsection{Robustness Under In-Context Learning (ICL)}


Previous study \cite{gu2023robustness} has shown that the LLMs are less sensitive to prompt / instruction variation when few-shot examples are provided in context.
While we are focused on zero-shot capabilities, for completeness, we re-ran all experiments in a few-shot setting.
%Although this is not the focus of our work, we have conducted all of our previous experiments equally under few-shot settings. 
We report these results in the \ref{section:icl_robustness}. The main finding is that while some discrepancy remains, in general ICL \textbf{slightly} decreases the sensitivity of models to the use of unobserved instructions.
This is intuitive, given that the examples themselves likely imply the desired task and may affect the distribution.

\begin{figure}
  \centering
  %\includegraphics[scale=0.17]{images/icl_full.pdf}
  \caption{The performance degradation when using unobserved instruction at \textsc{BBL} and \textsc{MMLU} with Flan-T5-XXL. We plot the accuracy degradation of all the unobserved instructions compared with the average accuracy of the observed ones. It could be seen that under one-shot in-context learning, the model is slightly more robust as the performance difference converges closer to 0}
  \label{fig:icl}
\end{figure}

\begin{comment}
\begin{figure*}[h]
  \begin{floatrow}
    \floatbox{figure}[.5\textwidth][\FBheight][t]{
      \centering
      \includegraphics[scale=0.16]{images/icl.pdf}
      \label{fig:perf-dist-reg}
    }
    {
      \caption{Plots of average degradations in performance versus the semantic distance while using unobserved instructions.}
    }
    % \label{fig:perf-dist-reg}
    \floatbox{table}[.5\textwidth][\FBheight][t]{
      \vspace{-10pt}
      \begin{tabular}{l c c c}
        \toprule
        \textbf{Dataset}     & \textbf{0-shot} & \textbf{1-shot} & $\Delta$ \textbf{Acc.} \\
        \midrule
        \textsc{Small (80M)} & 22.0            & -3.1            & -                      \\
        \textsc{Small (80M)} & 55.3            & -5.5            & -                      \\
        \textsc{Small (80M)} & 59.0            & -11.3           & -                      \\
        \textsc{Small (80M)} & 59.0            & -11.3           & -                      \\
        \textsc{Small (80M)} & 59.0            & -11.3           & -                      \\
        \bottomrule
      \end{tabular}
      \label{table:icl_improvement}
    }
    {
      \caption{Average degradations in performance for four categories. It could be seen that \textsc{MMLU} has the minimal distance and hence yields the least degradation}
    }
  \end{floatrow}
\end{figure*}
\end{comment}




\section{Aligning Equivalent Instructions}
\vspace{-0.5em}
\label{section:methods}

%As an attempt to address the issues discussed above, 
We now introduce a simple, lightweight, but effective method to improve the robustness of instruction-tuned LLMs.
The intuition is to introduce a term in the objective which explicitly encourages the model to yield similar predictions (and hence similar representations) for the same input when provided distinct but semantically equivalent instructions.


%\subsection{Aligning Instructions}

%For an autoregressive Language Model with
More specifically, we aim to align semantically equivalent instructions in the space induced by the model.
To this end we introduce soft embedding parameters with dimensions $\mathbb{R}^{d \times n}$; this is equivalent to adding $n$ novel tokens (with embedding dimension $d$) as prefixes to inputs (preceding instructions).
The intuition is to push the representations for semantically equivalent tasks close together.
To this end, we add additional term to the loss: The KL-divergence $\mathcal{L}_{\text{KL}}$ of the output probabilities between a reference instruction for a given task and paraphrased (semantically equivalent) version of the same.
We combine this with the standard cross-entropy loss, and fine-tune \emph{only} the introduced soft prompt parameters under this objective (Figure \ref{fig:method-schematic}).
Here $\lambda$ is a loss-weighting hyper-parameter, $\hat{y}^{(j)}_i$ and $\hat{y}_r^{(j)}$ are the distributions over the vocabulary $\mathcal{V}$ induced by the model with paraphrased instruction $i$ and the reference instruction $r$ at token position $j$.\footnote{We pad instances such that the lengths in a given batch are effectively equal; the sum is therefore from 1 to the length associated with the current batch, we omit this for simplicity.}

%observed instructions and train the prefix embedding to "drag" the unobserved and observed instruction closer together.
%, we introduce a new trainable parameter $\theta'\in \mathbb{R}^{d\times n}$ as the soft prefix embedding adding at the front of input instruction to "align" the unobserved instruction with the observed instruction. 

%To achieve the goal,
%we collect a set of paraphrases (see 5.2 for more details) of the 

%Hence, we freeze the model and fine-tune the additional parameters $\theta'$.

%We process each batch of training data %into $\{x_i, y_i\}_{|i=1...N+1}$, where $x_1=I(x)$ and $x_{i\neq1}=I'_{i-1}(x)$. 
%In other words,
%such that each batch comprises instances that share semantically equivalent instructions.
%One example (the first, arbitrarily) is always the original sample from the observed instruction tuning set. 
%first data is always the original sample from the observed training set.
%We then jointly optimize the soft embedding parameters with respect to a linear combination of KL-divergence (between output distributions for an instance given distinct but semantically equivalent instructions) and the standard cross-entropy loss for the corresponding target. 




%At each step during training, we optimize instances with paraphrased instructions and original instructions separately with different objectives. 
%For original instruction, we optimize it with the cross-entropy loss $\mathcal{L}_{CE}$ with the ground truth. For instance, with  paraphrased instructions, instead of learning the correct answer, we optimize the KL-divergence $\mathcal{L}_{KL}$ of the output probabilities between it and its observed counterpart. The overall loss of each training batch is given as follows:


\begin{figure}[h]
  \centering
  \begin{subfigure}[b]{0.6\textwidth}
    \centering
    %{\includegraphics[scale=0.365]{images/method-fig.pdf}}
  \end{subfigure}
  \hfill
  \begin{subfigure}[b]{0.125\textwidth}
    \begin{align*}
       & \mathcal{L} = (1-\lambda)\mathcal{L}_{\text{CE}} + \lambda\mathcal{L}_{\text{KL}}                              \\
       & \mathcal{L}_{\text{KL}} = \frac{1}{N-1}\sum_{i\neq r}^{N}\sum_j \text{KL}(\hat{y}_{i}^{(j)} ||\hat{y}_r^{(j)}) \\
       & \hat{y}_{i}^{(j)} = \text{Softmax}(p_{i}^{(j)})\text{, }p_{i}^{(j)}\in\mathbb{R}^{|\mathcal{V}|}               \\
    \end{align*}
  \end{subfigure}
  \caption{Schematic depiction of the proposed instruction alignment method (left) and associated loss terms (right). Dotted (red) lines indicate backpropagation; we update only the soft prompt parameters, which we show yields performance superior to fine-tuning all model parameters.
    \label{fig:method-schematic}}
\end{figure}

\begin{comment}
\begin{tabular}{p{4cm}c}
  {
    \begin{align*}
       & \mathcal{L} = (1-\lambda)\mathcal{L}_{\text{CE}}(\hat{y}_{1}, y_1) + \lambda\mathcal{L}_{\text{KL}}      \\
       & \mathcal{L}_{\text{KL}} = \frac{1}{N-1}\sum_{i=2}^{N}\sum \text{KL}(\hat{y}_{i}^{(j)} ||\hat{y}_1^{(j)}) \\
       & \hat{y}_{i}^{(j)} = \text{Softmax}(p_{i}^{(j)})\text{, }p_{i}^{(j)}\in\mathbb{R}^{|v|}                   \\
    \end{align*}
  } & {\includegraphics[scale=0.4]{images/method-fig.pdf}}
\end{tabular}
\end{comment}

%\subsection{Dataset Preparation}

\begin{comment}
\begin{table}%[h]
  \small
  \centering
  \begin{tabular}{c l c c c }
    \toprule
    \textbf{Collection} & \textbf{Task}                    & \textbf{Type} & \textbf{Instances} & \textbf{\# of Paraphrases} \\
    \midrule
    \textbf{NIV2}       & \textsc{CommonsenseQA}           & QA            & 120                & 14                         \\
    \textbf{NIV2}       & \textsc{AI2 ARC-Challenge}       & QA            & 120                & 14                         \\
    \textbf{NIV2}       & \textsc{OpenbookQA}              & QA            & 120                & 14                         \\
    \textbf{NIV2}       & \textsc{MathQA General}          & QA            & 120                & 14                         \\
    \textbf{NIV2}       & \textsc{XCSR MC Classification}  & MC            & 120                & 14                         \\
    \textbf{NIV2}       & \textsc{Newscomm Classification} & MC            & 120                & 14                         \\
    \textbf{NIV2}       & \textsc{TriviaQA Classification} & BC            & 120                & 14                         \\
    \textbf{NIV2}       & \textsc{MultiRC Classification}  & BC            & 120                & 14                         \\
    \hline
    \textbf{Alpaca}     & \textsc{Alpaca}                  & General       & 500                & 9                          \\
    \bottomrule
  \end{tabular}
  \vspace{-0.5em}
  \caption{Data used for soft prompt alignment; we paraphrase reference instructions using GPT-4.}
  \label{tab:data_stat}
\end{table}
\vspace{-0.5em}

\end{comment}

Optimizing for the above objective requires paraphrased instructions $i$ for each task in the training data; we generate these automatically as follows.
For instruction-tuning dataset, we sample a small amount of training data to use for alignment. %to serve as the instances for soft prompt alignment. 
We paraphrase these reference instructions using GPT-4. For the Alpaca collection, we randomly sampled 1000 tasks and paraphrased them with three prompts, and collected the top three candidates under temperature 0.5. For the Flan collection, we randomly sampled 986 instances from the mixture with 3 prompts with greedy decoding. % Table \ref{tab:data_stat} reports counts yielded for different tasks we assembled in this way. 

For fine-tuning, we then create instances for each example by pairing them with every distinct instruction available for the corresponding task.
We then form batches by including one instance featuring the original instruction and the rest comprising paraphrased instructions. For the implementation of the prefix, we follow the setting of \cite{li2021prefix}, which freezes the model parameters and just trains the prefix embeddings with the MLP layers.
%For each instruction (1 origin instruction + paraphrases) of the dataset, we combine it with all the instances. 
%Hence, for each instance with the original instruction $I$, we have N counterparts with paraphrased instruction $I'_{n}$ for $n\in \{1...N\}$.  


\section{Results}

We experiment with the proposed method using two representative instruction-tuned LLMs: Flan-XL (3B) and Alpaca (7B).
We compare the canonical versions of these models trained in the usual way (the same evaluated in Table \ref{tab:main_result}) to variants fine-tuned using our proposed approach.
We ablate components of our method to tease out the contributions of data and objectives.
%components of the proposed alignment strategy. 
%That is, we ablate the components of the proposed approach and report corresponding results to tease out the contributions of data and objectives. 

Specifically, we consider variants where we: Fine-tune all model parameters on the additional, automatically generated instruction paraphrases (FT); impose the new KL loss term (again fine-tuning all model parameters; FT+KL); introduce the additional soft prompt parameters and fine-tune on the paraphrase instances, but without KL (PT); and then the full proposed strategy, which introduces the soft prompt parameters and optimizes them for the loss augmented with the KL term ({\bf PT+KL}).

%We compare variants of these models fine-tuned on the corresponding instruction datasets in the usual way to results achieved 
%For fine-tuning the baseline, we compare the performance of fine-tuning and so. More implementation details can be seen in Appendix ?

\begin{table}[h]
  \centering
  \small
  \begin{tabular}{l c c c c c c }
    \toprule
                        & \multicolumn{3}{c}{\textsc{MMLU}}                      & \multicolumn{3}{c}{\textsc{BBL}}                                                                                                                                                                                                                                                           \\ [0.5ex]
    \midrule
    \textbf{Model}      & \textsc{Obs.}                                          & \textsc{Unobs.}                                        & Avg.                                                   & \textsc{Obs.}                                          & \textsc{Unobs.}                                        & Avg.                                                   \\
    \textsc{Flan-T5-3B} & 48.1                                                   & 47.5                                                   & 47.8                                                   & \textbf{56.1}                                          & 51.9                                                   & 54.0                                                   \\
    FT                  & 39.4 \textcolor{red}{\textbf{(-8.7)}}                  & 40.1 \textcolor{red}{\textbf{(-7.4)}}                  & 39.8 \textcolor{red}{\textbf{(-8.0)}}                  & 48.2 \textcolor{red}{\textbf{(-7.9)}}                  & 42.3 \textcolor{red}{\textbf{(-9.2)}}                  & 45.3 \textcolor{red}{\textbf{(-8.7)}}                  \\
    FT+KL               & 41.8 \textcolor{red}{\textbf{(-6.3)}}                  & 43.6 \textcolor{red}{\textbf{(-3.9)}}                  & 45.9 \textcolor{red}{\textbf{(-1.9)}}                  & 47.7 \textcolor{red}{\textbf{(-8.4)}}                  & 43.1 \textcolor{red}{\textbf{(-8.8)}}                  & 45.4 \textcolor{red}{\textbf{(-8.6)}}                  \\
    PT                  & 48.1 \textcolor{MidnightBlue}{\textbf{(+0.0)}}         & 47.6 \textcolor{ForestGreen}{\textbf{(+0.1)}}          & 47.9 \textcolor{ForestGreen}{\textbf{(+0.1)}}          & 55.9 \textcolor{red}{\textbf{(-0.2)}}                  & 52.1 \textcolor{ForestGreen}{\textbf{(+0.2)}}          & 54.0 \textcolor{MidnightBlue}{\textbf{(+0.0)}}         \\
    \textbf{PT+KL}      & \textbf{48.1} \textcolor{ForestGreen}{\textbf{(+0.1)}} & \textbf{47.9} \textcolor{ForestGreen}{\textbf{(+0.4)}} & \textbf{48.0} \textcolor{ForestGreen}{\textbf{(+0.2)}} & 55.9 \textcolor{red}{\textbf{(-0.2)}}                  & \textbf{53.7} \textcolor{ForestGreen}{\textbf{(+1.8)}} & \textbf{54.8} \textcolor{ForestGreen}{\textbf{(+0.8)}} \\
    \midrule
    \textsc{Alpaca-7B}  & 41.9                                                   & 39.7                                                   & 40.8                                                   & 47.6                                                   & 42.9                                                   & 45.3                                                   \\
    FT                  & 40.3 \textcolor{red}{\textbf{(-1.6)}}                  & 39.1 \textcolor{red}{\textbf{(-0.6)}}                  & 39.7 \textcolor{red}{\textbf{(-1.1)}}                  & 44.4 \textcolor{red}{\textbf{(-3.2)}}                  & 42.1 \textcolor{red}{\textbf{(-0.8)}}                  & 43.4 \textcolor{red}{\textbf{(-2.0)}}                  \\
    FT+KL               & 39.7 \textcolor{red}{\textbf{(-2.2)}}                  & 40.2 \textcolor{ForestGreen}{\textbf{(+0.5)}}          & 40.0 \textcolor{red}{\textbf{(-0.8)}}                  & 45.6 \textcolor{red}{\textbf{(-2.0)}}                  & 42.8 \textcolor{red}{\textbf{(-0.1)}}                  & 44.2 \textcolor{red}{\textbf{(-1.1)}}                  \\
    PT                  & 42.1 \textcolor{ForestGreen}{\textbf{(+0.2)}}          & 40.0 \textcolor{ForestGreen}{\textbf{(+0.3)}}          & 41.1 \textcolor{ForestGreen}{\textbf{(+0.3)}}          & 47.5 \textcolor{red}{\textbf{(-0.1)}}                  & 43.0 \textcolor{ForestGreen}{\textbf{(+0.1)}}          & 45.3 \textcolor{MidnightBlue}{\textbf{(+0.0)}}         \\
    \textbf{PT+KL}      & \textbf{42.4} \textcolor{ForestGreen}{\textbf{(+0.5)}} & \textbf{41.8} \textcolor{ForestGreen}{\textbf{(+2.1)}} & \textbf{42.1} \textcolor{ForestGreen}{\textbf{(+1.3)}} & \textbf{47.9} \textcolor{ForestGreen}{\textbf{(+0.3)}} & \textbf{46.6} \textcolor{ForestGreen}{\textbf{(+3.7)}} & \textbf{47.3} \textcolor{ForestGreen}{\textbf{(+2.0)}} \\
    \bottomrule
  \end{tabular}
  \caption{Results and ablations of the proposed soft prompt alignment method. All ablated versions use the augmented set with automatically paraphrased instructions. FT refers to simply fine-tuning (with teacher-forcing) on this additional data; PT denotes prefix tuning (i.e., introducing soft prompt parameters); KL refers to the alignment objective that we proposed above. Using all of these components together yields the best performance, especially on unobserved instructions.}
  \label{tab:alignment-results}
\end{table}

We report results in Table \ref{tab:alignment-results}.
Two observations: (1) The proposed soft prompt alignment strategy ({\bf PT+KL}) yields consistent improvements across the tasks and models considered and especially improves performance on unobserved instructions, as anticipated. (2) The full benefit of the approach is realized only when all components---the additional automatically paraphrased training instructions, soft prompt parameters, and additional KL loss term---are in place.

\begin{table*}[h]
  \small
  \centering
  \begin{tabular}{l l l l}
    \toprule
    \textbf{Dataset} & \textbf{Closest Distance Before} & \textbf{Closest Distance After} & \textbf{$\Delta$ Acc. Improvement} (\%) \\
    \midrule
    \textsc{MMLU}    & 22.2                             & 21.3                            & + 0.3\%                                 \\
    \midrule
    \textsc{BBL QA}  & 22.4                             & 23.0                            & + 0.4\%                                 \\
    \textsc{BBL BC}  & 30.1                             & \textbf{27.9}                   & \textbf{+ 4.2\%}                        \\
    \textsc{BBL MC}  & 26.0                             & 24.6                            & + 0.3\%                                 \\
    \bottomrule
  \end{tabular}
  \caption{Average distances before and after soft prompt alignment with Flan-T5-XL.} %Ten observed and unobserved instructions with 50 instances for each dataset.} %For each unobserved instruction, its closest observed instruction is found by the average $\ell$2 distances across instances. We report the result of all the datasets in the Appendix}
  \vspace{-0.5em}
  \label{tab:improvement_distance}
\end{table*}


Following our approach in \ref{section:mmlu_variance}, we take the average distance between observed and unobserved instructions before and after alignment.
Table \ref{tab:improvement_distance} shows that our method brings observed and unobserved instruction representations closer together.
The similarity is most increased in the case of the biggest accuracy gain, further suggesting the mechanism of improvement provided by soft prompt alignment.
%Further, a large reduction of distance is observed in most of the binary classification datasets, which is also where the improvement is the most significant.


%\subsection{Discussion}

\section{Conclusions}
\label{section:conclusions}
\vspace{-.35em}

Instruction-tuned LLMs have emerged as a promising means of achieving zero-shot performance with smaller models that is competitive to, and sometimes even better than, that observed using much larger LLMs \cite{longpre2023flan,alpaca}. 
In this work we empirically characterized the \emph{robustness} of such models with respect to instruction rephrasings. 
In particular, we collected manually composed instructions from 36 graduate students in NLP across 75 tasks, and we evaluated different families of instruction-tuned LLMs (Flan, Alpaca, and T0) when provided observed and unobserved instructions (seen in training and not, respectively). %(i.e., seen in training) and unobserved (manual instructions we collected).
We found that using the latter consistently degrades model performance, indicating that models are unduly sensitive to instruction phrasings.

We then proposed a simple mechanism intended to improve the robustness of instruction-tuned LLMs.
This approach entails introducing an additional loss term that penalizes the model for inducing dissimilar distributions over output tokens when using (a) paraphrased instructions as opposed to (b) a reference instruction for the same task.
We found that training under this objective consistently (though modestly) improves results, and in particular mitigates the degradation observed when previously unobserved instructions are used. 

\section{Limitations}
\label{section:limitations}
\vspace{-.35em}

This work has important limitations: For example we only evaluated ``mid-sized'' models (<20B parameters), it is unclear if our findings would generalize to much larger instruction-tuned models. (However, we note that instruction tuning has been most promising for smaller models.) 
We also restricted our evaluation to three task types: QA and multi-class and binary classification.

\vspace{0.1em}
\noindent{\bf Ethics} This work does not have an explicit ethical dimension, but we acknowledge that all LLMs are likely to encode problematic biases; it is unclear how instruction-tuning might interact with these. 

\section{Acknowledgments}

This work was supported by the National Science Foundation (NSF) grant 1901117.

We thank Jay DeYoung and Alberto Mario Ceballos Arroyo for their advice and feedback on the paper. 
We also thank Alberto Mario Ceballos Arroyo, Arnab Sen Sharma, Bowen Zhao, Eric Todd, Hanming Li, Hiba Ahsan, Hye Sun Yun, Shulin Cao, Jay DeYoung, Jered McInerney, Ji Qi, Jifan Yu, Jize Jiang, Kaisheng Zeng, Koyena Pal, Kundan Krishna, Linxiao Nie, Hailong Jin, Jinxin Matthew Liu, Millicent Li, Monica Munnangi, Nikhil Prakash, Pouya Pezeshpour, Sanjana Ramprasad, Sarthak Jain, Shangqing Tu, Somin Wadhwa, Tingjian Zhang, Hao Wesley Peng, Xiaozhi Wang, Xingyu Lu, Xin Lv, Zijun Yao for providing manually written instructions.





\section{Preamble}


Set the title and author using \verb|\title| and \verb|\author|. Within the author list, format multiple authors using \verb|\and| and \verb|\And| and \verb|\AND|; please see the \LaTeX{} source for examples.

By default, the box containing the title and author names is set to the minimum of 5 cm. If you need more space, include the following in the preamble:
\begin{quote}
  \begin{verbatim}
\setlength\titlebox{<dim>}
\end{verbatim}
\end{quote}
where \verb|<dim>| is replaced with a length. Do not set this length smaller than 5 cm.

\section{Document Body}

\subsection{Footnotes}

Footnotes are inserted with the \verb|\footnote| command.\footnote{This is a footnote.}

\subsection{Tables and figures}

See Table~\ref{tab:accents} for an example of a table and its caption.
\textbf{Do not override the default caption sizes.}

\begin{table}
  \centering
  \begin{tabular}{lc}
    \hline
    \textbf{Command} & \textbf{Output} \\
    \hline
    \verb|{\"a}|     & {\"a}           \\
    \verb|{\^e}|     & {\^e}           \\
    \verb|{\`i}|     & {\`i}           \\
    \verb|{\.I}|     & {\.I}           \\
    \verb|{\o}|      & {\o}            \\
    \verb|{\'u}|     & {\'u}           \\
    \verb|{\aa}|     & {\aa}           \\\hline
  \end{tabular}
  \begin{tabular}{lc}
    \hline
    \textbf{Command} & \textbf{Output} \\
    \hline
    \verb|{\c c}|    & {\c c}          \\
    \verb|{\u g}|    & {\u g}          \\
    \verb|{\l}|      & {\l}            \\
    \verb|{\~n}|     & {\~n}           \\
    \verb|{\H o}|    & {\H o}          \\
    \verb|{\v r}|    & {\v r}          \\
    \verb|{\ss}|     & {\ss}           \\
    \hline
  \end{tabular}
  \caption{Example commands for accented characters, to be used in, \emph{e.g.}, Bib\TeX{} entries.}
  \label{tab:accents}
\end{table}

\subsection{Hyperlinks}

Users of older versions of \LaTeX{} may encounter the following error during compilation:
\begin{quote}
  \tt\verb|\pdfendlink| ended up in different nesting level than \verb|\pdfstartlink|.
\end{quote}
This happens when pdf\LaTeX{} is used and a citation splits across a page boundary. The best way to fix this is to upgrade \LaTeX{} to 2018-12-01 or later.

\subsection{Citations}

\begin{table*}
  \centering
  \begin{tabular}{lll}
    \hline
    \textbf{Output}           & \textbf{natbib command} & \textbf{Old ACL-style command} \\
    \hline
    \citep{Gusfield:97}       & \verb|\citep|           & \verb|\cite|                   \\
    \citealp{Gusfield:97}     & \verb|\citealp|         & no equivalent                  \\
    \citet{Gusfield:97}       & \verb|\citet|           & \verb|\newcite|                \\
    \citeyearpar{Gusfield:97} & \verb|\citeyearpar|     & \verb|\shortcite|              \\
    \hline
  \end{tabular}
  \caption{\label{citation-guide}
    Citation commands supported by the style file.
    The style is based on the natbib package and supports all natbib citation commands.
    It also supports commands defined in previous ACL style files for compatibility.
  }
\end{table*}

Table~\ref{citation-guide} shows the syntax supported by the style files.
We encourage you to use the natbib styles.
You can use the command \verb|\citet| (cite in text) to get ``author (year)'' citations, like this citation to a paper by \citet{Gusfield:97}.
You can use the command \verb|\citep| (cite in parentheses) to get ``(author, year)'' citations \citep{Gusfield:97}.
You can use the command \verb|\citealp| (alternative cite without parentheses) to get ``author, year'' citations, which is useful for using citations within parentheses (e.g. \citealp{Gusfield:97}).

\subsection{References}

\nocite{Ando2005,andrew2007scalable,rasooli-tetrault-2015}

The \LaTeX{} and Bib\TeX{} style files provided roughly follow the American Psychological Association format.
If your own bib file is named \texttt{custom.bib}, then placing the following before any appendices in your \LaTeX{} file will generate the references section for you:
\begin{quote}
  \begin{verbatim}
\bibliography{custom}
\end{verbatim}
\end{quote}

You can obtain the complete ACL Anthology as a Bib\TeX{} file from \url{https://aclweb.org/anthology/anthology.bib.gz}.
To include both the Anthology and your own .bib file, use the following instead of the above.
\begin{quote}
  \begin{verbatim}
\bibliography{anthology,custom}
\end{verbatim}
\end{quote}

Please see Section~\ref{sec:bibtex} for information on preparing Bib\TeX{} files.

\subsection{Appendices}

Use \verb|\appendix| before any appendix section to switch the section numbering over to letters. See Appendix~\ref{sec:appendix} for an example.

\section{Bib\TeX{} Files}
\label{sec:bibtex}

Unicode cannot be used in Bib\TeX{} entries, and some ways of typing special characters can disrupt Bib\TeX's alphabetization. The recommended way of typing special characters is shown in Table~\ref{tab:accents}.

Please ensure that Bib\TeX{} records contain DOIs or URLs when possible, and for all the ACL materials that you reference.
Use the \verb|doi| field for DOIs and the \verb|url| field for URLs.
If a Bib\TeX{} entry has a URL or DOI field, the paper title in the references section will appear as a hyperlink to the paper, using the hyperref \LaTeX{} package.

\section*{Acknowledgements}

% Entries for the entire Anthology, followed by custom entries
\bibliography{anthology,custom}

\appendix

\section{Example Appendix}
\label{sec:appendix}

This is an appendix.

\end{document}
